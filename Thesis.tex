	%% ----------------------------------------------------------------
%% Thesis.tex -- MAIN FILE (the one that you compile with LaTeX)
%% ---------------------------------------------------------------- 

% Set up the document
\documentclass[letterpaper, 11pt, oneside]{Thesis}  % Use the "Thesis" style, based on the ECS Thesis style by Steve Gunn
\graphicspath{Figures/}  % Location of the graphics files (set up for graphics to be in PDF format)

% Include any extra LaTeX packages required
\usepackage[square, numbers, comma, sort&compress]{natbib}  % Use the "Natbib" style for the references in the Bibliography
\usepackage[spanish, es-tabla]{babel}
\usepackage[rightcaption]{sidecap}
\usepackage{verbatim}  % Needed for the "comment" environment to make LaTeX comments
\usepackage{vector}  % Allows "\bvec{}" and "\buvec{}" for "blackboard" style bold vectors in maths
\hypersetup{urlcolor=blue, colorlinks=true}  % Colours hyperlinks in blue, but this can be distracting if there are many links.

%% ----------------------------------------------------------------
\begin{document}
\frontmatter      % Begin Roman style (i, ii, iii, iv...) page numbering

% Set up the Title Page
\title  {Integrales de Trayectoria y Espacios Curvos}
\authors  {\texorpdfstring
            {\href{diegoa_baronm@hotmail.com}{Diego Alberto Barón Moreno}}
            {Author Name}
            }
\addresses  {\groupname\\\deptname\\\univname}  % Do not change this here, instead these must be set in the "Thesis.cls" file, please look through it instead
\date       {\today}
\subject    {}
\keywords   {}

\maketitle
%% ----------------------------------------------------------------

\setstretch{1.3}  % It is better to have smaller font and larger line spacing than the other way round

% Define the page headers using the FancyHdr package and set up for one-sided printing
\fancyhead{}  % Clears all page headers and footers
\rhead{\thepage}  % Sets the right side header to show the page number
\lhead{}  % Clears the left side page header

\pagestyle{fancy}  % Finally, use the "fancy" page style to implement the FancyHdr headers

%% ----------------------------------------------------------------
% Declaration Page required for the Thesis, your institution may give you a different text to place here
%\Declaration{

%\addtocontents{toc}{\vspace{1em}}  % Add a gap in the Contents, for aesthetics

%I, AUTHOR NAME, declare that this thesis titled, `THESIS TITLE' and the work presented in it are my own. I confirm that:

%\begin{itemize} 
%\item[\tiny{$\blacksquare$}] This work was done wholly or mainly while in candidature for a research degree at this University.
 
%\item[\tiny{$\blacksquare$}] Where any part of this thesis has previously been submitted for a degree or any other qualification at this University or any other institution, this has been clearly stated.
 
%\item[\tiny{$\blacksquare$}] Where I have consulted the published work of others, this is always clearly attributed.
 
%\item[\tiny{$\blacksquare$}] Where I have quoted from the work of others, the source is always given. With the exception of such quotations, this thesis is entirely my own work.
 
%\item[\tiny{$\blacksquare$}] I have acknowledged all main sources of help.
 
%\item[\tiny{$\blacksquare$}] Where the thesis is based on work done by myself jointly with %others, I have made clear exactly what was done by others and what I have contributed myself.
%\\
%\end{itemize}
 
 
%Signed:\\
%\rule[1em]{25em}{0.5pt}  % This prints a line for the signature
 
%Date:\\
%\rule[1em]{25em}{0.5pt}  % This prints a line to write the date
%}
%\clearpage  % Declaration ended, now start a new page
%
%% ----------------------------------------------------------------
% The "Funny Quote Page"
\pagestyle{empty}  % No headers or footers for the following pages

\null\vfill
% Now comes the "Funny Quote", written in italics
\textit{``Si ayudo a una sola persona a tener esperanza, no habré vivido en vano.''}

\begin{flushright}
Martin Luther King
\end{flushright}

\vfill\vfill\vfill\vfill\vfill\vfill\null
\clearpage  % Funny Quote page ended, start a new page
%% ----------------------------------------------------------------

% The Abstract Page
\addtotoc{Resumen}  % Add the "Abstract" page entry to the Contents
\abstract{
\addtocontents{toc}{\vspace{1em}}  % Add a gap in the Contents, for aesthetics

En este trabajo pretendo presentar de una manera organizada y concisa el compendio de los temas que he estudiado durante los últimos semestres de mi carrera, en partícular la temática tratada son las intregrales de trayectoria. El primer capítulo trata sobre las integrales de trayectoria en espacio plano (espacio-tiempo de curvatura nula) en este capítulo empiezo con la definción del propagador y la aplicación de este formalismo a problemas de primera cuantización, luego en lo que sigue se estudia el método conocido como álgebra de corrientes (un metodo basado en integrales de trayectoria) para estudiar los problemas de segunda cuantización, esto incluye campos escalares, campos fermiónicos y para finalizar el capítulo se hace una presentación de las teorías Gauge, los campos de Yang-Mills y el método de Faddeev-Popov.
\\
\\
Para el segundo capítulo los esfuerzos se centran en tratar el problema de la cuantización en espacios curvos, en la primera sección se realiza una presentación introductoria de teorías clásicas de campo en espacios curvos para luego estudiar integrales de trayectoria en problemas de primera cuantización en variedades generales. Este estudio incluye la deducción del propagador para la ecuación de Schrödinger generalizada, las transformaciones de coordenadas y la utilidad de este formalismo para resolver problemas de interés en mecánica cuántica, como el átomo de Hidrógeno, utilizando el método de integrales de trayectoria. En la sección final se ha incluído un aporte propio que creemos no está reportado en la literatura y es el cálculo del corrimiento de líneas de interferencia en el efecto Sagnac, calculado vía de integrales de trayectoria. 
}

\clearpage  % Abstract ended, start a new page
%% ----------------------------------------------------------------

\setstretch{1.3}  % Reset the line-spacing to 1.3 for body text (if it has changed)

% The Acknowledgements page, for thanking everyone
\acknowledgements{
\addtocontents{toc}{\vspace{1em}}  % Add a gap in the Contents, for aesthetics

En la gran cantidad de tiempo que ha pasado desde que comencé mis estudios he conocido a muchas buenas personas que han ayudado a que este trabajo sea posible, no solo desde un punto de vista académico, sino brindadome pequeños momentos fuera de la academia. Quiero agradecer a mi asesor Nelson Vanegas por haberme dado la oportunidad de trabajar bajo su guía, gracias al tiempo que dedicó a resolver mis inquietudes es que este trabajo es posible. Quiero hacer énfasis en el hecho de   

}
\clearpage  % End of the Acknowledgements
%% ----------------------------------------------------------------

\pagestyle{fancy}  %The page style headers have been "empty" all this time, now use the "fancy" headers as defined before to bring them back


%% ----------------------------------------------------------------
\renewcommand{\contentsname}%
    {Tabla de Contenido}%
	
\lhead{\emph{Contents}}  % Set the left side page header to "Contents"
\tableofcontents  % Write out the Table of Contents

%% ----------------------------------------------------------------

\lhead{\emph{Integrales de trayectoria y espacios curvos.}}  % Set the left side page header to "List if Figures"
\listoffigures  % Write out the List of Figures

%% ----------------------------------------------------------------
%\lhead{\emph{List of Tables}}  % Set the left side page header to "List of Tables"
%\listoftables  % Write out the List of Tables

%% ----------------------------------------------------------------
%\setstretch{1.5}  % Set the line spacing to 1.5, this makes the following tables easier to read
%\clearpage  % Start a new page
%\lhead{\emph{Abbreviations}}  % Set the left side page header to "Abbreviations"
%\listofsymbols{ll}  % Include a list of Abbreviations (a table of two columns)
%{
% \textbf{Acronym} & \textbf{W}hat (it) \textbf{S}tands \textbf{F}or \\
%\textbf{LAH} & \textbf{L}ist \textbf{A}bbreviations \textbf{H}ere \\

%}

%% ----------------------------------------------------------------
%\clearpage  % Start a new page
%\lhead{\emph{Physical Constants}}  % Set the left side page header to "Physical Constants"
%\listofconstants{lrcl}  % Include a list of Physical Constants (a four column table)
%{
% Constant Name & Symbol & = & Constant Value (with units) \\
%Speed of Light & $c$ & $=$ & $2.997\ 924\ 58\times10^{8}\ \mbox{ms}^{-\mbox{s}}$ (exact)\%\

%}

%% ----------------------------------------------------------------
%\clearpage  %Start a new page
%\lhead{\emph{Symbols}}  % Set the left side page header to "Symbols"
%\listofnomenclature{lll}  % Include a list of Symbols (a three column table)
%{
% symbol & name & unit \\
%$a$ & distance & m \\
%$P$ & power & W (Js$^{-1}$) \\
%& & \\ % Gap to separate the Roman symbols from the Greek
%$\omega$ & angular frequency & rads$^{-1}$ \\
%}
%% ----------------------------------------------------------------
% End of the pre-able, contents and lists of things
% Begin the Dedication page

\setstretch{1.3}  % Return the line spacing back to 1.3

\pagestyle{empty}  % Page style needs to be empty for this page
\dedicatory{Dedicado a mi madre\ldots Una luchadora.	}

\addtocontents{toc}{\vspace{2em}}  % Add a gap in the Contents, for aesthetics


%% ----------------------------------------------------------------
\mainmatter	  % Begin normal, numeric (1,2,3...) page numbering
\pagestyle{fancy}  % Return the page headers back to the "fancy" style

% Include the chapters of the thesis, as separate files
% Just uncomment the lines as you write the chapters

\chapter{Introducción}
En la actualidad la ciencia, especificamente la física, nos presenta un panorama en el cual podemos entender el mundo al nivel más fundamental de la siguiente manera: existen 3 tipos de interacciones fundamentales: la gravedad, la interacción electrodébil y la interacción fuerte. La gravedad es la interacción responsable de la aceleración que sufren los cuerpos debido a su contenido de masa-energía, la teoría de la relatividad general (RG) explica satisfactoriamente los fenómenos asociados a esta interacción, esta es una teoría clásica de campos donde las fuerzas gravitatorias son reemplazadas por el concepto de curvatura del espacio-tiempo [19] % Introduction

\chapter{Integrales de trayectoria	 en espacio plano.}
El formalismo mas común de la mecánica cuantica se deriva de cambiar las variables clasicas de posición y momentum ($p$ y $q$) por operadores que obedecen el álgebra:
\begin{equation}
[\hat{q},\hat{p}]=i\hbar
\end{equation}
Esta relación se conoce como la condción de cuantización de Heisenberg, en general en la mecánica cuantica de operadores estos últimos viven en un espacio de Hilbert.
\\
\\
La formulación de integrales de camino se basa en la noción de \textbf{propagador}, esta función es tal que dada una funcion de onda en un instante de tiempo $t_1$, $\psi(x_1,t_1)$ da la evolucion hasta un instante de tiempo $t_2$, entregando $\psi(x_2,t_2)$. En cierta manera es parecido al principio de Huygens:
\begin{equation}
\psi(x_f,t_f)=\int K(q_f,t_f;q_i,t_i)\psi(q_i,t_i)dq_i
\end{equation}
De acuerdo con la mecánica cuántica $\psi(q_f,t_f)$ representa la probabilidad de que una partícula se encuentre en un punto $q_f$ en el instante de tiempo $t_f$, por tanto $K(q_f,t_f;q_i,t_i)$ representa la amplitud de probabilidad de transición entre un estado ($q_i,t_i$) y ($q_f,t_f$).
\begin{equation}
P(q_f,t_f;q_i,t_i)=\Vert K(q_f,t_f;q_i,t_i) \Vert^2
\end{equation}
Si dividimos el intervalo de tiempo en $t_i\rightarrow t \rightarrow t_f$, tenemos de la definición de $K$:
\begin{eqnarray}
\nonumber \psi(q_f,t_f)&=&\int\int K(q_f,t_f;q,t)K(q,t;q_i,t_i)dqdq_i\\
\Rightarrow K(q_f,t_f;q_i,t_i)&=&\int dq K(q_f,t_f;q,t)K(q,t;q_i,t_i)
\end{eqnarray}
Como ejemplo de lo anterior podemos analizar el experimento de la doble rendija. En la Figura 1 encontramos un esquema de este:
\begin{figure}[h]
\centering
\includegraphics[width=9cm]{Imagenes/Fig1}
\caption[Esquema del experimento de la doble rendija]{Experimento de la doble rendija.}
\end{figure}
Si $K(2A,1)$ es la probabilidad de que un electrón pase por la rendija 2A, entonces podemos escribir:
\begin{equation}
K(3,1)=K(2A,1)K(3,2A)+K(2B,1)K(3,2B)
\end{equation}
Al tomar el módulo cuadrado de la expresión (2.5) se generarán los términos de interferencia necesarios para describir el patrón de difracción. No podemos decir que el electrón tomó un camino u el otro, de una manera más simple: este siguió todos los caminos posibles!

\section{La ecuación de Schrödinger.}
En el cuadro de Schrodinger la evolución de un sistema cuántico afecta al ket que representa al estado del sistema, la ecuación que rige la dinámica del mismo es la \textbf{Ecuación de Schrödinger}
\begin{equation}
i\hbar\frac{\partial}{\partial t}|\psi\rangle_S=\hat{H}|\psi\rangle_S
\end{equation}
Sabemos que $\psi(q,t)=\langle q|\psi \rangle_S$ donde $|q\rangle$ son autoestados de la posición, la relacion entre el cuadro de Heisenberg y el de Schrödinger es $|\psi\rangle_H=e^{iHt/\hbar}|\psi\rangle_S$. Si definimos $|qt\rangle \equiv e^{iHt/\hbar}|q\rangle$, entonces $\psi (q,t)=\langle qt|\psi\rangle_H$.

Vamos a mostrar que $K(q_f,t_f;q_i,t_i)=\left\langle q_ft_f|q_it_i \right\rangle$, la relación de completez nos permite escribir:
\begin{eqnarray}
\nonumber \langle q_ft_f|\psi\rangle &=& \int\langle q_ft_f|q_it_i\rangle\langle q_it_i|\psi\rangle dq_i\\
&=& \int \langle q_ft_f|q_it_i\rangle \psi(q_i,t_i)dq_i
\end{eqnarray}
Y comparando (2.7) y (2.2), vemos que:
\begin{equation}
K(q_f,t_f;q_i,t_i)=\langle q_ft_f|q_it_i\rangle
\end{equation}
Así, el propagador es proporcional a la amplitud de probabilidad de transición entre el estado inicial $|q_it_i\rangle$ y final $|q_ft_f\rangle$. La idea ahora es expresar el propagador como una integral de trayectoria. Partamos el intervalo temporal $(t_i,t_f)$ en $n+1$ piezas de igual duración $\tau$, así:
\begin{equation}
\langle q_ft_f|q_it_i\rangle \int dq_1dq_2...dq_n\langle q_ft_f|q_nt_n\rangle \langle q_nt_n|q_{n-1}t_{n-1}\rangle...\langle q_1t_1|q_it_i\rangle
\end{equation}
Calculemos uno de estos elementos:
\begin{eqnarray}
\nonumber \langle q_{j+1}t_{j+1}|q_jt_j\rangle &=&\langle q_{j+1}|e^{-iHt_{j+1}/\hbar}e^{iHt_j/\hbar}|q_j=\langle q_{j+1}|e^{-i\tau H/\hbar}|q_j\rangle \hspace{0.4cm}\text{A primer orden,}\\
\nonumber &=&\langle q_{j+1}|1-i\tau H/\hbar|q_j\rangle=\delta(q_{j+1}-q_j)-i\tau\hbar\langle q_{j+1}|H|q_j\rangle\\
&=& \frac{1}{2\pi\hbar}\int dp\ \text{Exp}[\frac{ip}{\hbar}(q_{j+1}-q_j)]-\frac{i\tau}{\hbar}\langle q_{j+1}|H|q_j\rangle 
\end{eqnarray}	
Si asumimos que el Hamiltoniano es una función de $p$ y $q$de la forma: $H=\frac{p^2}{2m}+V(q)$, entonces:
\begin{eqnarray}
\nonumber \langle q_{j+1}|\frac{p^2}{2m}|q_j \rangle &=& \int dpdp\prime \langle q_{j+1}|p\prime\rangle\langle p\prime|\frac{p^2}{2m}|p\rangle \langle p|q_j \rangle\\
\nonumber &=&\int \frac{dpdp\prime}{2\pi\hbar}\ \text{Exp}[\frac{i}{\hbar}(p\prime q_{j+1}-pq_j)]\frac{p^2}{2m}\delta(p-p\prime)\\
&=& \int \frac{dp}{2\pi\hbar}\ \text{Exp}[\frac{i}{\hbar}p(q_{j+1}-q_j)]\frac{p^2}{2m}
\end{eqnarray}
De una manera similar:
\begin{eqnarray}
\nonumber \langle q_{j+1}|V(q)|q_j \rangle &=& V(\frac{q_{j+1}+q_j}{2})\langle q_{j+1}|q_j\rangle\\
\nonumber &=& V(\frac{q_{j+1}+q_j}{2})\delta(q_{j+1}-q_j)\\
\langle q_{j+1}|V(q)|q_j \rangle &=& V(\frac{q_{j+1}+q_j}{2})\int \frac{dp}{2\pi\hbar}\ \text{Exp}[\frac{i}{\hbar}p(q_{j+1}-q_j)]
\end{eqnarray}
Por tanto $\langle q_{j+1}|H|q_j\rangle=\int \frac{dp}{h}\ \text{Exp}[\frac{i}{h}p(q_{j+1}-q_j)]H(p,q)$ y:
\begin{equation}
\langle q_{j+1}t_{j+1}|q_jt_j\rangle=\frac{1}{h}\int dp_j\ \text{Exp}[\frac{i}{\hbar}[p_j(q_{j+1}-q_j)-\tau H(p_j,q_j)]]
\end{equation}
Finalmente:
\begin{equation}
\langle q_{f}t_{f}|q_it_i\rangle=\lim_{N \to \infty}\int\prod_{j=1}^{N}dq_j\prod_{j=0}^{N}\frac{dp_j}{h}\ \text{Exp}\left\{ \frac{i}{\hbar}\sum_{j=0}^{N}[p_{j}(q_{j+1}-q_{j})-\tau H(p_{j},q_{j})]\right\}
\end{equation}
En el continuo:
\begin{equation}
\langle q_{f}t_{f}|q_it_i\rangle=\int\frac{\mathcal{D}q\mathcal{D}p}{h}\text{Exp\ensuremath{\left\{ \frac{i}{\hbar}\int_{t_{i}}^{t_{f}}(p\dot{q}-H(p,q))dt\right\} }}
\end{equation}
En el continuo $q$ se vuelve una función de $t$ y la integral anterior, es una integral funcional. La expresión (2.15) es la integral de trayectoria para la amplitud de transición entre $(q_i,t_i)$ y $(q_f,t_f)$. Esta integral se hace sobre todas las posibles trayectorias en el espacio de fase y $q(t), p(t)$ son funciones y no operadores, sin embargo es natural preguntarse por la convergencia de (2.15), esto es algo no trivial, sin embargo, de ahora en adelante asumiremos que esta integral existe y converge. 	
\\
Si el Hamiltoniano es tal que $H=\frac{p^2}{2m}+V(q)$
\begin{equation}
\langle q_{f}t_{f}|q_it_i\rangle=\lim_{N \to \infty}\int\prod_{j=1}^{N}dq_j\prod_{j=0}^{N}\frac{dp_j}{h}\ \text{Exp}\left\{ \frac{i}{\hbar}\sum_{j=0}^{N}[p_{j}(q_{j+1}-q_{j})-\tau \frac{p_j^2}{2m}-V(q)\tau]\right\}
\end{equation}
Y sabiendo que: $\int_{-\infty}^{\infty}\text{Exp}[-ax^{2}+bx+c]dx=\text{Exp}[\frac{b^{2}}{4a}+c](\frac{\pi}{a})^{\frac{1}{2}}$, entonces:
\begin{equation}
\langle q_{f}t_{f}|q_it_i\rangle=\lim_{N\to\infty}[\frac{1}{h}(\frac{\text{\ensuremath{\pi\hbar}2m}}{\tau})^{\frac{1}{2}}]^{N}\int\prod_{1}^{N}dq_{j}\text{Exp}\left\{ \frac{i}{\hbar}\sum_{0}^{N}\left((\frac{q_{j+1}-q_{j}}{\tau})^{2}\frac{m}{2}-V\right)\tau\right\} 
\end{equation}
En el continuo:
\begin{equation}
\langle q_{f}t_{f}|q_it_i\rangle=\mathcal{N}\int\mathcal{D}q\text{Exp\ensuremath{\left\{ \frac{i}{\hbar}\int_{t_{i}}^{t_{f}}\mathcal{L}(q,\dot{q})dt\right\} }}
\end{equation}
Donde $\mathcal{L}(q,\dot{q})$ es el lagrangiano clásico, sin embargo esto solo pasa si asumimos una forma específica del Hamiltoniano, cuando esto no es asi se tiene una acción efectiva. En teorñia de campos por ejemplo esta descomposición solo se puede hacer en el caso de campos Abelianos.
\newpage
 
 

\subsection{Teoría de Perturbaciones.}

Vamos a ilustrar cómo usar el método de integrales de trayectoria en procesos de dispersión, este tipo de procesos involucran la interacción de una partícula con un potencial $V(x)$. Debido a que no siempre podemos calcular analíticamente la integral (2.18) entonces es necesario acudir a la teoría de perturbaciones, esta es aplicable en el regimen en que la energía de interacción $E_I<\hbar$. En este caso podemos escribir:
\begin{equation}
\text{Exp}\left[\frac{-i}{\hbar}\int_{t_{i}}^{t_{f}}V(x,t)dt\right]=1-\frac{i}{\hbar}\int_{t_{i}}^{t_{f}}V(x,t)dt+\left(\frac{i}{\hbar}\right)^{2}\frac{1}{2!}\left[\int_{t_{i}}^{t_{f}}V(x,t)dt\right]^{2}+...
\end{equation}
Cuando reemplazamos esto en la expresión (2.18) obtenemos:
\begin{equation}
K=K_0+K_1+K_2+...\ \text{donde}\ K_0=\mathcal{N}\int \mathcal{D}x \text{Exp}\left[ \frac{i}{\hbar}\int \frac{1}{2}m\dot{x}^2dt\right] 
\end{equation}
Para calcular $K_0$, escribamos la forma discretizada de (2.20)
\begin{eqnarray}
\nonumber K_0&=&\lim_{n\to\infty}\left(\frac{m}{i\hbar\tau}\right)^{\left(\frac{n+1}{2}\right)}\int_{-\infty}^{\text{\ensuremath{\infty}}}\prod_{j=1}^{n}dx_{j}\text{Exp}\left[\frac{im}{2\hbar\tau}\sum_{j=0}^{n}\left(x_{j+1}-x_{j}\right)^{2}\right]\\
\nonumber &=& \lim_{n\to\infty}\left(\frac{m}{i\hbar\tau}\right)^{\left(\frac{n+1}{2}\right)}\frac{1}{\left(n+1\right)^{\frac{1}{2}}}\left(\frac{i\hbar\tau}{m}\right)^{\frac{n}{2}}\text{Exp}\left[\frac{im}{2\hbar(n+1)\tau}\left(x_{f}-x_{i}\right)^{2}\right]\\
K_0(x_ft_f,x_it_i)&=& \Theta(t_{f}-t_{i})\left(\frac{m}{i\hbar(t_{f}-t_{i})}\right)^{1/2}\text{Exp}\left[\frac{im}{2\hbar(t_{f}-t_{i})}\left(x_{f}-x_{i}\right)^{2}\right] 
\end{eqnarray}
Calculemos ahora $K_1$:
\begin{equation}
K_1=\lim_{n\to\infty}\frac{-i}{\hbar}N^{(n+1)/2}\sum_{i=1}^{n}\int\text{Exp}\left[\frac{im}{2\hbar\tau}\sum_{j=0}^{n}(x_{j+1}-x_{j})^{2}\right]V(x_{i},t_{i})dx_{1}...dx_{n}
\end{equation}
Ahora como solo $V(x)$ depende de $x_i$, separamos (2.22) como:
\begin{eqnarray}
\nonumber K_1&=&\lim_{n\to\infty}\frac{-i}{\hbar}\sum_{i=1}^{n}\left\{ \int N^{(n-i+1)/2}\text{Exp}\left[\frac{im}{2\hbar\tau}\sum_{j=i}^{n}(x_{j+1}-x_{j})^{2}\right]dx_{i}...dx_{n}\right\}\\
& &\times  \left\{ \int N^{i/2}\text{Exp}\left[\frac{im}{2\hbar\tau}\sum_{j=0}^{i-1}(x_{j+1}-x_{j})^{2}\right]dx_{1}...dx_{i-1}\right\} V(x_{i},t_{i})
\end{eqnarray}
Los dos términos en corchetes en (2.23) son $K_0(xt,x_ft_f)$ y $K_0(x_it_i,xt)$, así:
\begin{equation}
K_1(x_ft_f,x_it_i)= -\frac{i}{\hbar}\int_{t_{i}}^{t_{f}}dt\int_{-\infty}^{\infty}K_{0}(x_{f}t_{f}.xt)V(x,t)K_{0}(xt,x_{i}t_{i})dx
\end{equation}
Como $K_0(x_ft_f,xt)=0$ si $t>t_f$ y $K_0(xt,x_it_i)=0$ si $t<t_i$, entonces podemos escribir:
\begin{equation}
K_1(x_ft_f,x_it_i)=-\frac{i}{\hbar}\int_{-\infty}^{\infty}dt\int_{-\infty}^{\infty}K_{0}(x_{f}t_{f}.xt)V(x,t)K_{0}(xt,x_{i}t_{i})dx 
\end{equation}
De la misma manera se puede probar que:
\begin{equation}
K_2(x_ft_f,x_it_i)=\left(\frac{-i}{\hbar}\right)^2\int_{-\infty}^{\infty}dt_{1}dt_{2}dx_{1}dx_{2}K_{0}(x_{f}t_{f}.x_{2}t_{2})V_2K_{0}(x_{2}t_{2},x_{1}t_{1})V_1K_{0}(x_{1}t_{1},x_{i}t_{i})
\end{equation}
Esta es una solución en series para $K$ y recibe el nombre de Serie de Born, en la expresión general para $K_n$ no se tiene el factor $n!$ ya que hay ese mismo numero de formas para ordenar los $n$ potenciales $V(x)$ que entran en el propagador.
\\
\\
Por último mostraremos que \textbf{el propagador es la función de Green de la ecuación de Schrödinger.} Para esto sustituyamos la expresión para la serie de Born en la ecuación (2.2):
\begin{eqnarray}
\nonumber \psi(\vec{x}_f,t_f)&=&\int K_{0}(\vec{x_{f}}t_{f},\vec{x_{i}}t_{i})\psi(\vec{x_{i}},t_{i})d\vec{x_{i}}\\
& & \frac{-i}{\hbar}\int K_{0}(\vec{x_{f}}t_{f},\vec{x}t)V(\vec{x},t)K_{0}(\vec{x}t,\vec{x_{i}}t_{i})\psi(\vec{x_{i}},t_{i})d\vec{x}dtd\vec{x_{i}}+O(\hbar^{2})
\end{eqnarray}
Hemos cambiado a tres dimensiones espaciales y los otros términos en la serie hacen converger el último $K_0$ a $K$, por tanto:
\begin{equation}
\psi(\vec{x}_f,t_f)=\int K_{0}(\vec{x_{f}}t_{f},\vec{x_{i}}t_{i})\psi(\vec{x_{i}},t_{i})d\vec{x_{i}}-\frac{i}{\hbar}\int K_{0}(\vec{x_{f}}t_{f},\vec{x}t)V(\vec{x},t)\psi(\vec{x},t)d\vec{x}dt
\end{equation}
Cuando $t_i \to -\infty$,no hay presencia de potencial por tanto $\psi$ se vuelve una onda plana. Así:
\begin{equation}
\psi(\vec{x}_f,t_f)=\phi(\vec{x}_ft_f)-\frac{i}{\hbar}\int K_{0}(\vec{x_{f}}t_{f},\vec{x}t)V(\vec{x},t)\psi(\vec{x},t)d\vec{x}dt
\end{equation}
Aplicando el operador $\hat{H}=\frac{\hbar^2}{2m}\nabla_{\vec{x}_ft_f}+i\hbar\frac{\partial}{\partial t_f}$ en la ecuación (2.29) y usando $\hat{H}\psi(x,t)=V(x,t)\psi(x,t)$:
\begin{eqnarray}
\nonumber \hat{H}(\psi(\vec{x}_f,t_f))&=&\hat{H}(\phi(\vec{x}_ft_f))-\frac{i}{\hbar}\int\hat{H} (K_{0}(\vec{x_{f}}t_{f},\vec{x}t))V(\vec{x},t)\psi(\vec{x},t)d\vec{x}dt\\
\nonumber V(\vec{x}_f,t_t)\psi(\vec{x}_f,t_f)&=&0-\frac{i}{\hbar}\int\hat{H} (K_{0}(\vec{x_{f}}t_{f},\vec{x}t))V(\vec{x},t)\psi(\vec{x},t)d\vec{x}dt
\end{eqnarray}
Por tanto:
\begin{equation}
\left(\frac{\hbar^2}{2m}\nabla_{\vec{x}_ft_f}+i\hbar\frac{\partial}{\partial t_f}\right)K_{0}(\vec{x_{f}}t_{f},\vec{x}t)=i\hbar\delta(\vec{x}-\vec{x}_f)\delta(t-t_f)
\end{equation}
Esto último era lo que queríamos probar.

\subsection{Matriz $\mathcal{S}$.}
En un experimento de dispersión es razonable suponer que las partículas al principio y al final del proceso son partículas libres, estas ondas planas están distribuidas en todo el espacio. Sin embargo esto último lleva a una contradicción ya que la presencia del centro dispersor no permite que en sus cercanias la solución sea una onda plana. Para resolver este inconveniente se puede proponer lo que se llama una fuente dinámica, que se "prenda y apague" lentamente tal que las partículas en los estados (final/incial) puedan ser libres y por tanto la aproximación de ondas planas sea válida en este regimen.\\



\section{El experimento de la doble rendija.}

\section{Campos escalares.}
\section{Campos fermiónicos.}
\section{Teorías Gauge y campos de Yang-Mills}
\section{La teoria de Yukawa} % Background Theory 

\chapter{Integrales de trayectoria en espacios curvos.}


	   	
\section{Teoría clásica de campos en espacios curvos.}
Al nivel más fundamental la situación que nos presenta la fisica teórica es un modelo, con un alto grado de soporte experimental, en el cual solo hay tres interacciones: la cromo-dinámica cuaántica (QCD), que explica la interacción mediante la cual los quarks interactúan para formar los hadrones; la interacción débil (EW) que explica los procesos de decaimiento radiactivo y los fenomenos electromagneticos en una teoría unificada; y finalmente la gravedad. Las primeras dos interacciones se entienden en el contexto de la teoria cuántica de campos, en particular la teoría de campos Gauge, donde las interacciones se explican mediante el intercambio de bosones Gauge (gluones para QCD y el fotón y los bosones W,Z para EW). Sin embargo la teoría que mejor explica los fenómenos gravitatorios. la relativdad general (GR), es una teoría totalmente diferente debido a su naturaleza geométrica.
\\
\\
Sin embargo el objetivo de la física es reducir el numero de teorías, conceptos y esquemas al mínimo, es por esto que muchos físicos trabajan en teorías que unifiquen estos dos esquemas: supergravedad, teoría de supercuerdas, gravedad cuántica de lazos, etc. Debido a que las otras dos interacciones fundamentales se entienden en términos de teorías cuánticas de campo, ¿no deberia darsele a la gravedad un tratamiento cuántico?
\\
Por tanto, en una busqueda de teorías más alla de GR, ¿hacia dónde nos dirigimos?. Es bien sabido que no se puede hacer con las otras fuerzas fundamentales lo mismo que con la gravedad, en la teoría gravitatoria uno cambia la aceleración producida por la fuerza gravitacional por un sistema de referencia acelerado y esto se puede hacer debido a la equivalencia entre masa inercial y masa gravitatoria. Sin embargo esto no se puede hacer con las fuerzas fundamentales restantes, por ejemplo, la aceleración que sufre una partícula cargada electricamente es proporcional a su carga e inversamente proporcional a su masa inercial ($a\propto \frac{q}{m}$). Como esta razón no es la misma para todas las partículas uno no puede encontrar un sistema de referencia donde globalmente "desaparezca" la fuerza eléctrica. Sin embargo en los 60`s se dieron cuenta que al \textit{calibrar} (traducción del ingles "gauging", refiríendose a transformaciones Gauge como transformaciones de calibre) la simetría de Lorentz, uno termina con una teoría muy parecida a la GR. En esta primera sección introduciremos esta idea, sin embargo antes daremos un ejemplo de teorías Gauge abelianas (electromagnetismo) y no-abelianas (campos de Yang Mills).
\subsection{Caso abeliano.}
Consideremeos un campo escalar complejo que obedece la ecuacion de K-G y tiene una densidad Lagrangiana:
\begin{equation}
\mathcal{L}=\partial_{\mu}\phi\partial^{\mu}\phi^{*}+m^{2}\phi\phi^{*}
\end{equation}
Esta densidad lagrangiana posee una simetría bago $\phi\to e^{-i\Lambda}\phi,\ \ \phi^{*}\to e^{i\Lambda}\phi^{*}$ donde $\Lambda$ es un parámetro constante, vemos que esta transformación deja a $\mathcal{L}$ invariante es decir $\delta\mathcal{L}$. Esta transformación es llamada transformación Gauge del primer tipo. En la versión infinitesimal de la transformación tenemos:
\begin{equation}
\delta\phi=-i\Lambda\phi,\ \ \delta\phi^{*}=i\Lambda\phi^{*},\ \ \delta(\partial_{\mu}\phi)=-i\Lambda\partial_{\mu}\phi,\ \ \delta(\partial_{\mu}\phi^{*})=i\Lambda\partial_{\mu}\phi^{*}
\end{equation}
Con esto definimos la corriente $j^{\mu}$ como:
\begin{eqnarray}
\nonumber \Lambda j^{\mu}&=&\frac{\partial\mathcal{L}}{\partial(\partial_{\mu}\phi)}(\delta\phi)+\frac{\partial\mathcal{L}}{\partial(\partial_{\mu}\phi^{*})}(\delta\phi^{*})=(\partial_{\mu}\phi^{*})(-i\Lambda\phi)+(\partial_{\mu}\phi)(i\Lambda\phi^{*})\\
j^\mu &=& i((\partial_{\mu}\phi)\phi^{*}-\phi(\partial_{\mu}\phi^{*}))
\end{eqnarray}
De donde
\begin{eqnarray}
\partial_\mu j^\mu &=& 0\\
\partial_{0}j^{0}=\partial_{i}j^{i}&\Rightarrow &\partial_{0}\int_{V}j^{0}dV=\int_{V}\nabla\cdot\vec{j}dV=\int_{\partial V}\vec{n}\cdot\vec{j}dS=0
\end{eqnarray} 
Por tanto
\begin{equation}
\partial_{0}\int_{V}j^{0}dV=0\Rightarrow\frac{dQ}{dt}=0;\ \ Q=\int_{V}j^{0}dV
\end{equation}
Identificando $Q$ con la carga eléctrica tenemos que el campo $\phi$ porta una carga eléctrica la cual es conservada. Sin embargo la transformación anterior demanda que $\phi$ cambie la cantidad indicada al mismo tiempo en todos los puntos del espacio-tiempo, este tipo de transformación no tiene el espíritu de la relatividad, la transformación deberá ser \textit{local} en vez de \textit{globar}, es decir, el parámetro $\Lambda$ debe depender de las coordenadas $\Lambda=\Lambda(x^\mu)$. Así:
\begin{equation}
\phi(x)\to e^{-i\Lambda(x)}\phi(x),\ \ \phi(x)^{*}\to e^{i\Lambda(x)}\phi(x)^{*}
\end{equation}
Esta es llamada una transformación Gauge del segundo tipo. En la versión infinitesimal:
\begin{eqnarray}
\nonumber \delta\phi=-i\Lambda(x)\phi &;&\ \ \delta\phi^{*}=i\Lambda(x)\phi^{*}\\
\delta(\partial_{\mu}\phi)=-i(\Lambda\partial_{\mu}\phi+\phi\partial_{\mu}\Lambda)&;&\ \ \delta(\partial_{\mu}\phi^{*})=i(\Lambda\partial_{\mu}\phi^{*}+\phi^{*}\partial_{\mu}\Lambda)
\end{eqnarray}
Por tanto el lagrangiano no es invariante
\begin{equation}
\delta\mathcal{L}=-i\phi\partial_{\mu}\Lambda\partial_{\mu}\phi^{*}+i\phi^{*}\partial_{\mu}\Lambda\partial_{\mu}\phi=(\partial_{\mu}\Lambda)j^{\mu}
\end{equation}
Para mantener el lagrangiano invariante debemos agregar un nuevo campo a $\mathcal{L}$, de tal manera que $\mathcal{L}_1=-ej^\mu A_\mu$. Y pedir que bajo una transformación local 
\begin{equation}
A_\mu \to A_\mu +\frac{1}{e}\partial_\mu\Lambda
\end{equation} 
Así:
\begin{eqnarray}
\nonumber \delta\mathcal{L}_{1}&=&-e(\delta j^{\mu})A_{\mu}-j^{\mu}\partial_{\mu}\Lambda\\
\delta\mathcal{L}+\delta\mathcal{L}_{1}&=&-e(\delta j^{\mu})A_{\mu}=-2eA^{\mu}\partial_{\mu}\Lambda(\phi\phi^{*})
\end{eqnarray}
Y si agregamos $\mathcal{L}_2=e^2\phi\phi^{*}A_\mu A^\mu\Rightarrow \delta\mathcal{L}_2=2eA^{\mu}\partial_{\mu}\Lambda(\phi\phi^{*})$, por tanto:
\begin{equation}
\delta\mathcal{L}+\delta\mathcal{L}_1+\delta\mathcal{L}_2=0
\end{equation}
Al introducir el campo $ A^\mu$ debemos introducir un término diferente del acople con el campo escalar, este término debe ser el que de las ecuaciones de movimiento de $ A^\mu$ en ausencia de fuentes. Si definimos $F^{\mu\nu}=\partial^{\mu}A^{\nu}-\partial^{\nu}A^{\mu}$, bajo la transformación Gauge $\delta(\partial^{\mu}A^{\nu})=\frac{1}{e}\partial_{\mu}\partial_{\nu}\Lambda\Rightarrow\delta F^{\mu\nu}=0$. Entonces el lagrangiano $\mathcal{L}_{3}=\frac{1}{4}F^{\mu\nu}F_{\mu\nu}$ es invariante bajo la transformación gauge local $\mathcal{L}_3=0$. Finalmente tenemos:
\begin{eqnarray}
\nonumber\mathcal{L}_{\text{total}}&=&\mathcal{L}+\mathcal{L}_1+\mathcal{L}_2+\mathcal{L}_3\\
\nonumber &=& (\partial_{\mu}\phi+ieA_{\mu}\phi)(\partial^{\mu}\phi^{*}-ieA^{\mu}\phi^{*})+m^{2}\phi\phi^{*}+\frac{1}{4}F^{\mu\nu}F_{\mu\nu}\\
&=& D_{\mu}\phi D^{\mu}\phi^{*}m^{2}\phi\phi^{*}+\frac{1}{4}F^{\mu\nu}F_{\mu\nu}
\end{eqnarray}
Habiendo definido la derivada covariante como $D_{\mu}\phi=\partial_{\mu}\phi+ieA_{\mu}\phi$. Ahora generalizemos la corriente $j^\mu$ a una corriente $J^\mu$ que dependa de las derivadas covariantes, así:\begin{eqnarray}
\nonumber \Lambda J^{\mu}&=&\frac{\partial\mathcal{L}}{\partial(D_{\mu}\phi)}(\delta\phi)+\frac{\partial\mathcal{L}}{\partial(D_{\mu}\phi^{*})}(\delta\phi^{*})=(D_{\mu}\phi^{*})(-i\Lambda\phi)+(D_{\mu}\phi)(i\Lambda\phi^{*})\\
J^\mu &=& i((D_{\mu}\phi)\phi^{*}-\phi(D_{\mu}\phi^{*}))
\end{eqnarray}
Esta corriente es conservada, las ecuaciones de movimiento de $A_\mu$ son:
\begin{equation}
\partial_{\nu}F^{\nu\mu}=-eJ^{\mu}\Rightarrow\partial_{\mu}J^{\mu}=\partial_{\mu}\partial_{\nu}F^{\nu\mu}=0
\end{equation} 
Esto último debido a que el tensor $F^{\nu\mu}$ es antisimétrico.
\\
\\
Recordemos ahora que en GR el tensor de curvatura es proporcional al conmutador de las derivadas covariantes, es decir, $[\nabla_{\mu},\nabla_{\nu}]e_{k}=R_{k\mu\nu}^{\rho}e_{\rho}$. Si notamos que $[D_{\mu},D_{\nu}]\phi=ie(\partial_{\mu}A_{\nu}-\partial_{\nu}A_{\mu})=ieF_{\mu\nu}\phi$, podemos establecer una analogía entre la curvatura y la intesidad del campo Gauge. Otro hecho que podemos notar es que recordando la definición del tensor de Rienmann [4] $R_{***}^{*}=\partial[_{*}\Gamma_{*}]+[\Gamma,\Gamma] $ y también que $F_{**}=\partial[_{*}A_{*}]$ se puede asociar el potencial $A$ en electrodinámica con el coeficiente de conexión $\Gamma$ en GR. En el caso  del campo electromagnético el segundo término de la definición del tensor de Rienmann no tiene un análogo, pero veremos como esta situación cambia cuando generalicemos al caso de una simetría Gauge local no abeliana. Para terminar esta sección escribamos el análisis anterior de una forma más compacta, consideremos una transformación:
\begin{equation}
\phi\to U\phi,\ \ \phi^{*}\to U^{\dagger}\phi^{*}\text{con}\ \ U=e^{i\Lambda}\Rightarrow UU^{\dagger}=1
\end{equation}
Bajo una transformación Gauge local tenemos $\partial_\mu U=i\partial_\mu\Lambda U$, así la transformación del campo $A_\mu$ es:
\begin{equation}
A_{\mu}\to A_{\mu}-\frac{i}{e}U^{\dagger}\partial_{\mu}U\Rightarrow F_{\mu\nu}\to F_{\mu\nu}
\end{equation}
Si dos matrices $U_1,U_2$ obedecen la última relación de la ecuación (3.16) entonces $U^{\prime}=U_1U_2$ también la obedece, si la cada matriz $U$ tiene un inverso $U^{-1}$ entonces este grupo de matrices forman un grupo. En particular este grupo es llamado el grupo de matrices unitarias de dimensión 1 (U(1)). Este es el grupo de simetría de la electrodinámica.



\subsection{Caso No-abeliano.}
Sería interesante extender la simetría U(1) para la electrodinámica solo por ver "que sucede", pero de hecho hay una motivación física para hacerlo. La primera vez que fue hecho esto sucedió en el contexto de la física nuclear y la idea original es de Yang y Mills [5]. La propuesta trataba sobre una cantidad física que hoy en día se conoce como el isospín. Estos dos físicos se dieron cuenta que había una relación entre la carga eléctrica de las partículas y la tercera componente de su isospín, asi si pensamos el isopín como la consecuencia de la simetría bajo rotación en un espacio abstracto ¿no es factible entonces entender esta cantidad extendiendo el grupo gauge de U(1) a SU(2)?
\\
\\
El caso de SU(2) es un poco diferente, ya que este es un grupo no abeliano. SU(2) es elgripo especial unitario de dimensión 2, un elemento del grupo es tal que: 
\begin{equation}
U=\left(\begin{array}{cc}
a & b\\
-b^{*} & a^{*}
\end{array}\right),\ |a|^{2}+|b|^{2}=1,\text{\ U}^{\dagger}U=1=UU^{\dagger}
\end{equation}
Podemos escribir un elemento general de SU(2) como: 
\begin{equation}
U=\text{Exp}\left(\frac{i}{2}\vec{\eta}\cdot\vec{\sigma}_\alpha\right)
\end{equation}
Donde $\vec{\sigma}$ son las matrices de Pauli:
\begin{equation}
\sigma_{x}=\left(\begin{array}{cc}
0 & 1\\
1 & 0
\end{array}\right);\ \sigma_{y}=\left(\begin{array}{cc}
0 & -i\\
i & 0
\end{array}\right);\ \sigma_{z}=\left(\begin{array}{cc}
1 & 0\\
0 & -1
\end{array}\right)
\end{equation}
Estas obedecen el álgebra $[\frac{\sigma_i}{2},\frac{\sigma_j}{2}]=\frac{i}{2}\epsilon_{ijk}\sigma_k$.
Introduciendo el doblete $\phi=\left(\begin{array}{c}
\phi_{1}\\
\phi_{2}
\end{array}\right);\ \phi^{\dagger}=\left(\begin{array}{cc}
\phi_{1}^{*} & \phi_{2}^{*}\end{array}\right)$ de SU(2) la densidad lagrangiana:
\begin{equation}
\mathcal{L}=\partial_{\mu}\phi\partial^{\mu}\phi^{\dagger}+m^{2}\phi\phi^{\dagger}
\end{equation}
Es invariante bajo transformaciones de SU(2):$\phi\to U\phi;\ \phi^\dagger\to\phi^\dagger U^\dagger$, si los parámetros de la transformación son independientes de las coordenadas (transformación Gauge del primer tipo), la corriente conservada es:
\begin{equation}
\delta\phi=\frac{i}{2}\sigma^{a}\alpha^{a}\phi,\ \ \delta\phi^{\dagger}=\frac{-i}{2}\sigma^{a}\alpha^{a}\phi^{\dagger},\ \ \delta(\partial_{\mu}\phi)=\frac{i}{2}\sigma^{a}\alpha^{a}\partial_{\mu}\phi,\ \ \delta(\partial_{\mu}\phi^{\dagger})=\frac{-i}{2}\sigma^{a}\alpha^{a}\partial_{\mu}\phi^{\dagger}
\end{equation}
Así:
\begin{eqnarray}
\nonumber \alpha^{a}j_{a}^{\mu}&=&\frac{\partial\mathcal{L}}{\partial(\partial_{\mu}\phi)}(\delta\phi)+\frac{\partial\mathcal{L}}{\partial(\partial_{\mu}\phi^{\dagger})}(\delta\phi^{\dagger})\\
j^{a}_{\mu}&=& \frac{i}{2}[\partial_{\mu}\phi^{\dagger}\sigma^{a}\phi-\phi^{\dagger}\sigma^{a}\partial_{\mu}\phi]
\end{eqnarray}
Esta corriente tiene cuatro componentes espaciales y tres internas de isospín. También, debido a que $(\square-m^2)(\phi,\phi^\dagger)=0\Rightarrow \partial_\mu j^{\mu}_{a}=0$. Ahora veamos que pasa si los parámetros $\alpha^a$ dependen de las coordenadas. Si $U=\text{Exp}\left(\frac{i}{2}\sigma^a\alpha^a(x)\right)$,tenemos que:
\begin{equation}
\delta\phi(x)=\frac{i}{2}\sigma^{a}(x)\alpha^{a}\phi(x)
\end{equation}
Para que la densidad lagrangiana (3.21) sea invariante SU(2), entonces los términos con la derivada del campo deben transformar de la misma forma que el campo en sí mismo. Así definimos la derivdada covariante tal que esta transforme como:
\begin{equation}
\delta\phi_{;\mu}=\frac{i}{2}g\sigma^{a}\alpha^{a}\phi_{;\mu}
\end{equation}  
\\

Esto se cumple si $\phi_{;\mu}\equiv D_{\mu}\phi=\phi_{,\mu}-igA_{\mu}^{a}\frac{\sigma^{a}}{2}\phi$ y $A_{\mu}^{a}$ transforma como $A_{\mu}^{a}\to A_{\mu}^{a}+\partial_{\mu}\alpha^{a}-g\epsilon_{abc}\alpha^{b}A_{\mu}^{c}$, veamos:
\begin{eqnarray}
\nonumber \delta\phi_{;\mu}&=&\frac{i}{2}g\sigma^{a}\alpha^{a}(\phi_{,\mu}-igA_{\mu}^{b}\frac{\sigma^{b}}{2}\phi)=\frac{i}{2}g\sigma^{a}\alpha^{a}\phi_{,\mu}-\frac{i^{2}g^{2}}{4}\sigma^{a}\alpha^{a}A_{\mu}^{b}\sigma^{b}\phi\\
&=&\frac{i}{2}g\sigma^{a}\alpha^{a}\phi_{,\mu}+\frac{g^{2}}{4}\alpha^{a}A_{\mu}^{b}\phi+\frac{i}{2}g^{2}\epsilon_{abc}\sigma^{a}A_{\mu}^{b}\alpha^{c}\phi
\end{eqnarray}
Por otro lado:
\begin{eqnarray}
\nonumber \delta\phi_{;\mu}&=&\delta\phi_{,\mu}-ig[\delta(A_{\mu}^{b})\frac{\sigma^{b}}{2}\phi+A_{\mu}^{b}\frac{\sigma^{b}}{2}\delta\phi]\\
&=&\frac{i}{2}g\sigma^{a}\alpha^{a}\phi_{,\mu}+\frac{g^{2}}{4}\alpha^{a}A_{\mu}^{b}\phi+\frac{i}{2}g^{2}\epsilon_{abc}\sigma^{a}A_{\mu}^{b}\alpha^{c}\phi 
\end{eqnarray}
Observando que las ecuaciones (3.26) y (3.27) son las mismas podemos concluir que la definición de la dereviada covariante es válida. Mostremos ahora que la transformación del campo Gauge se puede derivar de la definición de como actúa una transformación Gauge del segundo tipo en un campo vectorial, sabemos que:
\begin{equation}
A_\mu \to UA_\mu U^\dagger -\frac{i}{g}(\partial_\mu U)U^\dagger 
\end{equation}
Para una transformación infinitesimal:
\begin{eqnarray}
\nonumber UA_\mu U^{\dagger}&=&\left(1+\frac{i}{2}g\frac{\sigma^{a}}{2}\alpha^{a}\right)A_{\mu}^{b}\frac{\sigma^{b}}{2}(1-\frac{i}{2}g\frac{\sigma^{c}}{2}\alpha^{c})\\
\nonumber &=&A_\mu+\frac{ig}{4}A_{\mu}^{b}(\sigma^{a}\sigma^{b}\alpha^{a}-\sigma^{b}\sigma^{a}\alpha^{a})=A_{\mu}+\frac{ig}{4}A_{\mu}^{b}[\sigma^{a},\sigma^{b}]\alpha^{a}\\
&=&A_{\mu}-gA_{\mu}^{b}\epsilon_{abc}\frac{\sigma^{c}}{2}\alpha^{a}
\end{eqnarray}
Y 
\begin{equation}
(\partial_\mu U)U^\dagger=\frac{ig}{2}\sigma^a\partial_\mu \alpha^a
\end{equation}
Por tanto si reemplazamos (3.29) y (3.30) en (3.28) obtenemos:
\begin{equation}
A_{\mu}^{a}\to A_{\mu}^{a}+\partial_{\mu}\alpha^{a}-g\epsilon_{abc}\alpha^{b}A_{\mu}^{c}
\end{equation}
Ahora generalizando el tensor de intensidad como $F_{\mu\nu}=\frac{i}{g}[D_\mu,D_\nu]A$
\begin{equation}
F_{\mu\nu}=\partial_{\mu}A_{\nu}-\partial_{\nu}A_{\mu}-ig[A_{\mu},A_{\nu}]\Rightarrow F_{\mu\nu}^{a}=\partial_{\mu}A_{\nu}^{a}-\partial_{\nu}A_{\mu}^{a}+g\epsilon_{abc}A_{\mu}^{b}A_{\nu}^{c}
\end{equation}
Así en el caso no-abeliano el tensor de intensidad $F_{\mu\nu}$ ha adquirido in término cuadrático en $A_\mu$. Veamos que el tensor $F_{\mu\nu}$ NO es invariante Gauge:
\begin{eqnarray}
\nonumber F_{\mu\nu}^{\prime a}&=&\partial_{\mu}A_{\nu}^{\prime a}-\partial_{\nu}A_{\mu}^{\prime a}+g\epsilon_{abc}A_{\mu}^{\prime b}A_{\nu}^{\prime c}\\
\nonumber &=& \partial_{\mu}UA_{\nu}^{a}U^{-1}-\partial_{\nu}UA_{\mu}^{a}U^{-1}-\frac{i}{g}\partial_{\mu}[(\partial_{\nu}U)U^{-1}]+\frac{i}{g}\partial_{\nu}[(\partial_{\mu}U)U^{-1}]\\
\nonumber && +g\epsilon_{abc}UA_{\mu}^{b}A_{\nu}^{c}U^{-1}-\frac{i}{g}(\partial_{\mu}U)U^{-1}UA_{\nu}U^{-1}-\frac{i}{g}UA_{\mu}U^{-1}(\partial_{\nu}U)U^{-1}\\
&=& U[\partial_{\mu}A_{\nu}^{a}-\partial_{\nu}A_{\mu}^{a}+g\epsilon_{abc}A_{\mu}^{b}A_{\nu}^{c}]U^{-1}=UF_{\mu\nu}^{a}U^{-1}
\end{eqnarray}
Por tanto $F_{\mu\nu}$ NO es invariante Gauge y tampoco $F_{\mu\nu}F^{\mu\nu}$ ya que:
\begin{equation}
F_{\mu\nu}F^{\mu\nu}\to UF_{\mu\nu}F^{\mu\nu} U^{-1}
\end{equation}
Sin embargo la traza de esta cantidad es invariante, veamos:
\begin{equation}
Tr[F_{\mu\nu}F^{\mu\nu}]= Tr[UF_{\mu\nu}F^{\mu\nu} U^{-1}]=Tr[U^{-1}UF_{\mu\nu}F^{\mu\nu}]=Tr[F_{\mu\nu}F^{\mu\nu}]
\end{equation}
Lo último debido a la propiedad de ciclicidad de la traza, por tanto con todo lo anterior la densidad lagrangiana invariante de SU(2) es:
\begin{equation}
\mathcal{L}=D_{\mu}\phi^{\dagger}D^{\mu}\phi+m^{2}\phi^{\dagger}\phi-\frac{1}{2}Tr[F^{\mu\nu}F_{\mu\nu}]
\end{equation}








\subsection{La simetría de Lorentz como teoría Gauge.}
En las primeras dos secciones hemos tratado de hacer invariante Gauge local un campo escalar, sin embargo en lo que sigue de aquí en adelante consideraremos un campo espinorial, estos campos satisfacen la famosa ecuación de Dirac:
\begin{equation}
(i\gamma^\mu\partial_\mu-m)\psi=0
\end{equation}		
Donde las matrices de Dirac. $\gamma^\mu$ satisfacen el álgebra $\{\gamma^\mu,\gamma^\nu\}=2g^{\mu\nu}$. Una transformación de Lorentz infinitesimal para un campo escalar es de la forma 
\begin{equation}
\phi^\prime(x)=\left(1-\frac{i}{2}\omega_{\mu\nu}J^{\mu\nu}\right)\phi(x);\ \ J_{\mu\nu}=-i(x_\mu\partial_\nu-x_\nu\partial_\mu) 
\end{equation} 
Sin embargo para campos espinoriales hay una contribución matricial adicional a $J_{\mu\nu}$, la cual corresponde al espín intrínseco del campo, asi:
\begin{equation}
J_{\mu\nu}=L_{\mu\nu}+\Sigma_{\mu\nu};\ \ \Sigma_{\mu\nu}=\frac{i}{4}[\gamma_\mu,\gamma_\nu]
\end{equation}
Ahora para hacer la ecuación (3.37) invariante \textit{local} de Lorentz debemos hacer los parámetros de la transformación dependientes de las coordenadas e intentar introducir una derivada covariante que deje la ecuación de Dirac invariante Guage.

Pero primero debemos hacer un alto en el camino y hacernos la siguiente pregunta: ¿cómo vamos a tratar los campos espinoriales en RG? En particular como vamos a tratar las derivadas covariantes sobre ellos. Nosotros estamos familiarizados con la construcción de derivadas covariantes para vectores y tensores de diferente rango, pero, ¿qué hay de los espinores? Esta no es una pregunta trivial y esto recae en el hecho de que el grupo de transformaciones generales de coordenadas. que se encuentra bajo los cimientos de la RG, tiene representaciones vectoriales y tensoriales, más no representaciones de espín.
\\
\\
Una situación similar ocurre para las rotaciones y las transformaciones de Lorentz. Las rotaciones están descritas por el grupo SO(3) del cual hay representaciones vectoriales y tensoriales, pero no espinoriales. Sin embargo el grupo SU(2) que es homeomórfico a SO(3) si tiene representación espinorial. Sin embargo no hay tal solución en RG, el grupo de transformaciones generales de coordenadas es un grupo de infinitos parámetros y no hay forma de encontrar un grupo similar con representación espinorial. Asi que de nuevo, volviendo a la pregunta ¿ cómo tratamos los espinores en RG? La respuesta fue dada por primera vez por Weyl [6] y en lo que sigue de aquí en adelante seguiremos un argumento que va en la misma dirección del argumento de Weyl en 1929. 

Sea $\mathcal{M}$ una variedad Lorentziana 4-dimensional, los \textit{vierbein} están definidos como: 
\begin{equation}
g_{\mu\nu}=e_{\mu}^{\alpha}e_{\nu}^{\beta}\eta_{\alpha\beta}
\end{equation}
Estos a su vez definen un marco ortogonal
\begin{equation}
\hat{\theta^{\alpha}}=e_{\mu}^{\alpha}dx^{\mu}
\end{equation}
Vamos a diferenciar los indices del marco ortogonal ($\alpha ,\beta ,\gamma ...$) y los indices de espacio-tiempo (coordenados) ($\mu ,\nu ,\lambda ...$). Con respecto al marco ortogonal local las matrices de Dirac $\gamma^\alpha=e^{\alpha}_{\mu}\gamma^\mu$ satisfacen el álgebra $\{\gamma^{\alpha},\gamma^{\beta}\}=2\eta^{\alpha\beta}$ y a su vez definen las correspondientes contrapartes en el espacio curvo mediante $\gamma^{\mu}=e_{\alpha}^{\mu}\gamma^{\alpha}$.
\\
\\
En el marco local (Espacio de Minkowski) un espinor transforma como:
\begin{equation}
\psi\to\rho(\Lambda)\psi \ \ ;\ \ \bar{\psi}\to\bar{\psi}\rho^{-1}(\Lambda)
\end{equation}
Donde $\rho(\Lambda)$ es la representación espinorial de las transformaciones de Lorentz.
\begin{equation}
\rho(\Lambda)=\text{Exp}\left[\frac{i}{2}\varepsilon^{\alpha\beta}\Sigma_{\alpha\beta}\right]\ \ ;\ \ \Sigma_{\alpha\beta}=\frac{i}{4}[\gamma_{\alpha},\gamma_{\beta}]
\end{equation}
Si por otro lado queremos que el lagrangiano de Dirac
\begin{equation}
\mathcal{L}_D=i\bar{\psi}\gamma^{\alpha}\nabla_{\alpha}\psi+m\bar{\psi}\psi
\end{equation}
Sea invariante bajo transformaciones \textit{locales} de Lorentz debemos pedir que:
\begin{equation}
\nabla_{\alpha}\psi\to\rho(\Lambda)\Lambda_{\alpha}^{\beta}\nabla_{\beta}\psi
\end{equation}
Es decir que la derivada covariante transforme como un espinor y a la vez como un vector de Lorentz, así:
\begin{eqnarray}
\nonumber \mathcal{L}_{D}^{\prime}=\bar{\psi^{\prime}}\{i\gamma^{\prime\alpha}\nabla_{\alpha}^{\prime}+m\}\psi^{\prime}&=& 	\bar{\psi}\{i\Lambda_{\delta}^{\alpha}e_{\mu}^{\delta}\gamma^{\mu}\Lambda_{\alpha}^{\beta}\nabla_{\beta}+m\}\psi\\
\nonumber &=& \bar{\psi}\{ie_{\mu}^{\delta}\gamma^{\mu}\Lambda_{\delta}^{\alpha}\eta^{\alpha h}\Lambda_{h}^{\beta}\nabla_{\beta}+m\}\psi;\ \ \text{Usando }\Lambda_{\delta}^{\alpha}\eta^{\alpha h}\Lambda_{h}^{\beta}=\eta^{\delta h}\\
&=& \bar{\psi}\{ie_{\mu}^{\delta}\gamma^{\mu}\eta^{\delta h}\nabla_{\beta}+m\}\psi=\bar{\psi}\{i\gamma^{\delta}\nabla_{\delta}+m\}\psi
\end{eqnarray}
Vamos a encontrar una expresión para la derivada covariante $\nabla_\alpha$, si suponemos que:
\begin{equation}
\nabla_{\alpha}\psi=e_{\alpha}^{\mu}[\partial_{\mu}+\Omega_{\mu}]\psi
\end{equation}
Bajo una transformación local la ecuación (3.47) queda:
\begin{eqnarray}
\nonumber \nabla_{\alpha}^{\prime}\psi^{\prime}&=& e_{\alpha}^{\prime\mu}[\partial_{\mu}+\Omega_{\mu}^{\prime}]\psi^{\prime}\\
\nonumber  \rho(\Lambda)\Lambda_{\alpha}^{\beta}\nabla_{\beta}\psi &=&\Lambda_{\alpha}^{\beta}e_{\alpha}^{\mu}[\partial_{\mu}+\Omega_{\mu}^{\prime}]\rho(\Lambda)\psi\\
\nonumber \rho(\Lambda)\Lambda_{\alpha}^{\beta}e_{\beta}^{\mu}[\partial_{\mu}+\Omega_{\mu}]\psi &=& \Lambda_{\alpha}^{\beta}e_{\alpha}^{\mu}[\partial_{\mu}+\Omega_{\mu}^{\prime}]\rho(\Lambda)\psi\\
\Rightarrow\Omega_{\mu}^{\prime}&=&\rho(\Lambda)\Omega_{\mu}\rho^{-1}(\Lambda)-(\partial_{\mu}\rho(\Lambda))\rho^{-1}(\Lambda)
\end{eqnarray}
Para encontrar la forma explícita de $\Omega_\mu$ consideremos una transformación infinitesimal de Lorentz.
\begin{equation}
\rho(\Lambda)\approx1+\frac{i}{2}\varepsilon^{\alpha\beta}\Sigma_{\alpha\beta}
\end{equation} 
En la ecuación (3.49) $\Sigma_{\alpha\beta}$ son los generadores del grupo de Lorentz en su representación espinorial , estos satisfacen el álgebra:
\begin{equation}
i[\Sigma_{\alpha\beta},\Sigma_{\gamma\delta}]=\eta_{\gamma\beta}\Sigma_{\alpha\delta}-\eta_{\gamma\alpha}\Sigma_{\beta\delta}+\eta_{\delta\beta}\Sigma_{\gamma\alpha}-\eta_{\delta\alpha}\Sigma_{\gamma\beta}
\end{equation}
Así bajo la transformación infinitesimal (3.49):
\begin{eqnarray}
\nonumber \Omega_{\mu}&\to &\left(1+\frac{i}{2}\varepsilon^{\alpha\beta}\Sigma_{\alpha\beta}\right)\Omega_{\mu}\left(1-\frac{i}{2}\varepsilon^{\alpha\beta}\Sigma_{\alpha\beta}\right)-\left(1-\frac{i}{2}\varepsilon^{\alpha\beta}\Sigma_{\alpha\beta}\right)\partial_{\mu}\left(1+\frac{i}{2}\varepsilon^{\alpha\beta}\Sigma_{\alpha\beta}\right)\\
&\to &\Omega_{\mu}+\frac{i}{2}\varepsilon^{\alpha\beta}[\Sigma_{\alpha\beta},\Omega_{\mu}]-\partial_{\mu}\frac{i}{2}\varepsilon^{\alpha\beta}\Sigma_{\alpha\beta}
\end{eqnarray}
Recordando que la 1-forma conexión transforma bajo transformaciones locales de Lorentz [4] como:
\begin{equation}
\omega_{\beta}^{\alpha}\to\omega_{\beta}^{\alpha}+\varepsilon_{\gamma}^{\alpha}\omega_{\beta}^{\gamma}-\omega_{\gamma}^{\alpha}\varepsilon_{\beta}^{\gamma}-d\varepsilon_{\beta}^{\alpha} \ ; \ \omega_{\beta}^{\alpha}\equiv\Gamma_{\gamma\beta}^{\alpha}\hat{\theta^{\gamma}}
\end{equation}
Al observar las ecuaciones (3.51) y (3.52) nos podemos hacer una guía de lo que debería ser una propuesta para $\Omega_\mu$, así:
\begin{equation}
\Omega_{\mu}=\frac{i}{2}\Gamma_{\mu}^{\alpha\beta}\Sigma_{\alpha\beta}
\end{equation} 
La propuesta de la ecuación (3.53) cumple con la regla de transformación (3.51). Así encontramos que la densidad lagrangiana que es invariante bajo transformaciones de coordenadas y transformaciones de Lorentz locales es:
\begin{equation}
\mathcal{L}_{D}=\bar{\psi}\{i\gamma^{\alpha}e_{\alpha}^{\mu}(\partial_{\mu}+\frac{i}{2}\Gamma_{\mu}^{\alpha\beta}\Sigma_{\alpha\beta})+m\}\psi
\end{equation}


\subsection{La ecuación de Dirac en el espacio de Schwarzschild.}

\section{La ecuación de Schrodinger en espacio curvo.}

\section{Transformaciones de coordenadas.}
\section{El átomo de Hidrógeno vía integrales de trayectoria.}
\section{El propagador para el efecto Sagnac.} % Experimental Setup

\chapter{Bibliografía.}
\begin{itemize}
\item[[1]] C. Cohen-Tannudji, B. Diu ;\textit{Quantum Mechanics, Volume One}; Wiley, New-York, (1991).
\item[[2]] Mathieu Beau; \textit{Feynman Integral and one/two slits electrons diffraction : an analytic study}; \texttt{arXiv:1110.2346v3}.
\item[[3]]  L.D. Faddeev,  V.N. Popov ;\textit{Feynman diagrams for the Yang-Mills field}; Physics Letters B Volume 25, Issue 1, 24 July 1967, Pages 29-30.
\item[[4]] Lewis H. Ryder;\textit{Quantum Field Theory};Second Edition (1996) 
\item[[5]] M Nakahara;\textit{Geometry, Topology and Physics};Second Edition (2003) 
\item[[6]]  C.N Yang,  R.L Mills ;\textit{Conservation of Isotopic Spin and Isotopic Gauge Invariance}; Phys. Rev. 96, 191 – Published 1 October 1954.
\item[[7]]  Hermann Weyl;\textit{Elektron und Gravitation. I}; Zeitschrift für Physik, May 1929, Volume 56, Issue 5, pp 330-352.
\item[[8]] R.M. Wald.;\textit{General Relativity.};University of Chicago Press, (1984). 
\item[[9]]  Bryce DeWitt;\textit{Dynamical Theory in Curved Spaces. I. A Review of the Classical and Quantum Action Principles.}; Reviews of Modern Physics, July 1957 .
\end{itemize}
 % Experiment 1

%\chapter*{Bibliografía.}
\begin{itemize}
\item[[1]] Lewis Ryder ;\textit{Introduction to General Relativity.};Cambridge University Press (2009).
\item[[2]] Steven Weinberg ;\textit{The Quantum Theory of Fields, Volume I-II.};Cambridge University Press (1995).
\item[[3]] Daniel Baumann;\textit{TASI Lectures on Inflation.};\texttt{https://arxiv.org/pdf/0907.5424v2.pdf}.
\item[[4]] C. Cohen-Tannudji, B. Diu ;\textit{Quantum Mechanics, Volume One}; Wiley, New-York, (1991).
\item[[5]] Mathieu Beau; \textit{Feynman Integral and one/two slits electrons diffraction : an analytic study}; \texttt{arXiv:1110.2346v3}.
\item[[6]]  L.D. Faddeev,  V.N. Popov ;\textit{Feynman diagrams for the Yang-Mills field}; Physics Letters B Volume 25, Issue 1, 24 July 1967, Pages 29-30.
\item[[7]] Lewis H. Ryder;\textit{Quantum Field Theory};Second Edition (1996) 
\item[[8]] M Nakahara;\textit{Geometry, Topology and Physics};Second Edition (2003) 
\item[[9]]  C.N Yang,  R.L Mills ;\textit{Conservation of Isotopic Spin and Isotopic Gauge Invariance}; Phys. Rev. 96, 191 – Published 1 October 1954.
\item[[10]]  Hermann Weyl;\textit{Elektron und Gravitation. I}; Zeitschrift für Physik, May 1929, Volume 56, Issue 5, pp 330-352.
\item[[11]] R.M. Wald.;\textit{General Relativity.};University of Chicago Press, (1984). 
\item[[12]]  Bryce DeWitt;\textit{Dynamical Theory in Curved Spaces. I. A Review of the Classical and Quantum Action Principles.}; Reviews of Modern Physics, July 1957 .
\item[[13]]  Christian Grosche;\textit{An introduction into the Feynman path integral.}; arXiv:hep-th/9302097v1 20 Feb 1993 .
\item[[14]]   Duru and H. Kleinert;\textit{Solution of the Path Integral for the H-Atom.}; Phys. Letters B 84, 185 (1979).
\item[[15]]   Duru and H. Kleinert;\textit{Quantum Mechanics of H-Atom from Path Integrals.};  Fortschr. d. Phys. 30, 401 (1982).
\item[[16]]  Hagen Kleinert;\textit{Path Integrals in Quantum Mechanics, Statistics, Polymer Physics, and Financial Markets.};  World Scientific, Singapore 2009.
\item[[17]]  M Chaichian and A Demichev;\textit{Path Integrals in Physics, Volume I
Stochastic Processes and Quantum Mechanics.};  IOP Publishing Ltd 2001.
\item[[18]] Dennis V. Perepelitsa;\textit{Path Integrals in Quantum Mechanics.};\texttt{http://goo.gl/y29SeM}.
\item[[19]] Sagnac, Georges;\textit{L'éther lumineux démontré par l'effet du vent relatif d'éther dans un interféromètre en rotation uniforme.}; Comptes Rendus 157: 708–710.
\item[[20]] Sagnac, Georges;\textit{Sur la preuve de la réalité de l'éther lumineux par l'expérience de l'interférographe tournant.};  Comptes Rendus 157: 1410–1413.
\item[[21]] Guido Rizzi, Matteo Luca Ruggiero ;\textit{Path Integrals in Quantum Mechanics.};Springer Science+Business Media Dordrecht (2004).
\item [[22]] Christopher C. Bernido and Glenn Aguarte; \textit{Summation over histories for a particle in spherical orbit around a black hole.};  Physical Review D, Volume 56, Number 4, (1997).
\item[[23]] Hector Alzate ;\textit{Física de las ondas.}; Univerisad de Antioquia (2007).
\end{itemize}
 % Experiment 2

%\input{Chapters/Chapter6} % Results and Discussion

%\input{Chapters/Chapter7} % Conclusion

%% ----------------------------------------------------------------
% Now begin the Appendices, including them as separate files

\addtocontents{toc}{\vspace{2em}} % Add a gap in the Contents, for aesthetics

\appendix % Cue to tell LaTeX that the following 'chapters' are Appendices

%\input{Appendices/AppendixA}	% Appendix Title

%\input{Appendices/AppendixB} % Appendix Title

%\input{Appendices/AppendixC} % Appendix Title

\addtocontents{toc}{\vspace{2em}}  % Add a gap in the Contents, for aesthetics
\backmatter

%% ----------------------------------------------------------------
\label{Bibliography}
\lhead{\emph{Bibliography}}  % Change the left side page header to "Bibliography"
\bibliographystyle{unsrtnat}  % Use the "unsrtnat" BibTeX style for formatting the Bibliography
\bibliography{Bibliography}  % The references (bibliography) information are stored in the file named "Bibliography.bib"

\end{document}  % The End
%% ----------------------------------------------------------------