\chapter{Integrales de trayectoria	 en espacio plano.}
El formalismo mas común de la mecánica cuantica se deriva de cambiar las variables clasicas de posición y momentum ($p$ y $q$) por operadores que obedecen el álgebra:
\begin{equation}
[\hat{q},\hat{p}]=i\hbar
\end{equation}
Esta relación se conoce como la condción de cuantización de Heisenberg, en general en la mecánica cuantica de operadores estos últimos viven en un espacio de Hilbert.
\\
\\
La formulación de integrales de camino se basa en la noción de \textbf{propagador}, esta función es tal que dada una funcion de onda en un instante de tiempo $t_1$, $\psi(x_1,t_1)$ da la evolucion hasta un instante de tiempo $t_2$, entregando $\psi(x_2,t_2)$. En cierta manera es parecido al principio de Huygens:
\begin{equation}
\psi(x_f,t_f)=\int K(q_f,t_f;q_i,t_i)\psi(q_i,t_i)dq_i
\end{equation}
De acuerdo con la mecánica cuántica $\psi(q_f,t_f)$ representa la probabilidad de que una partícula se encuentre en un punto $q_f$ en el instante de tiempo $t_f$, por tanto $K(q_f,t_f;q_i,t_i)$ representa la amplitud de probabilidad de transición entre un estado ($q_i,t_i$) y ($q_f,t_f$).
\begin{equation}
P(q_f,t_f;q_i,t_i)=\Vert K(q_f,t_f;q_i,t_i) \Vert^2
\end{equation}
Si dividimos el intervalo de tiempo en $t_i\rightarrow t \rightarrow t_f$, tenemos de la definición de $K$:
\begin{eqnarray}
\nonumber \psi(q_f,t_f)&=&\int\int K(q_f,t_f;q,t)K(q,t;q_i,t_i)dqdq_i\\
\Rightarrow K(q_f,t_f;q_i,t_i)&=&\int dq K(q_f,t_f;q,t)K(q,t;q_i,t_i)
\end{eqnarray}
Como ejemplo de lo anterior podemos analizar el experimento de la doble rendija. En la Figura 1 encontramos un esquema de este:
\begin{figure}[h]
\centering
\includegraphics[width=9cm]{Imagenes/Fig1}
\caption[Esquema del experimento de la doble rendija]{Experimento de la doble rendija.}
\end{figure}
Si $K(2A,1)$ es la probabilidad de que un electrón pase por la rendija 2A, entonces podemos escribir:
\begin{equation}
K(3,1)=K(2A,1)K(3,2A)+K(2B,1)K(3,2B)
\end{equation}
Al tomar el módulo cuadrado de la expresión (2.5) se generarán los términos de interferencia necesarios para describir el patrón de difracción. No podemos decir que el electrón tomó un camino u el otro, de una manera más simple: este siguió todos los caminos posibles!

\section{La ecuación de Schrödinger.}
En el cuadro de Schrodinger la evolución de un sistema cuántico afecta al ket que representa al estado del sistema, la ecuación que rige la dinámica del mismo es la \textbf{Ecuación de Schrödinger}
\begin{equation}
i\hbar\frac{\partial}{\partial t}|\psi\rangle_S=\hat{H}|\psi\rangle_S
\end{equation}
Sabemos que $\psi(q,t)=\langle q|\psi \rangle_S$ donde $|q\rangle$ son autoestados de la posición, la relacion entre el cuadro de Heisenberg y el de Schrödinger es $|\psi\rangle_H=e^{iHt/\hbar}|\psi\rangle_S$. Si definimos $|qt\rangle \equiv e^{iHt/\hbar}|q\rangle$, entonces $\psi (q,t)=\langle qt|\psi\rangle_H$.

Vamos a mostrar que $K(q_f,t_f;q_i,t_i)=\left\langle q_ft_f|q_it_i \right\rangle$, la relación de completez nos permite escribir:
\begin{eqnarray}
\nonumber \langle q_ft_f|\psi\rangle &=& \int\langle q_ft_f|q_it_i\rangle\langle q_it_i|\psi\rangle dq_i\\
&=& \int \langle q_ft_f|q_it_i\rangle \psi(q_i,t_i)dq_i
\end{eqnarray}
Y comparando (2.7) y (2.2), vemos que:
\begin{equation}
K(q_f,t_f;q_i,t_i)=\langle q_ft_f|q_it_i\rangle
\end{equation}
Así, el propagador es proporcional a la amplitud de probabilidad de transición entre el estado inicial $|q_it_i\rangle$ y final $|q_ft_f\rangle$. La idea ahora es expresar el propagador como una integral de trayectoria. Partamos el intervalo temporal $(t_i,t_f)$ en $n+1$ piezas de igual duración $\tau$, así:
\begin{equation}
\langle q_ft_f|q_it_i\rangle \int dq_1dq_2...dq_n\langle q_ft_f|q_nt_n\rangle \langle q_nt_n|q_{n-1}t_{n-1}\rangle...\langle q_1t_1|q_it_i\rangle
\end{equation}
Calculemos uno de estos elementos:
\begin{eqnarray}
\nonumber \langle q_{j+1}t_{j+1}|q_jt_j\rangle &=&\langle q_{j+1}|e^{-iHt_{j+1}/\hbar}e^{iHt_j/\hbar}|q_j=\langle q_{j+1}|e^{-i\tau H/\hbar}|q_j\rangle \hspace{0.4cm}\text{A primer orden,}\\
\nonumber &=&\langle q_{j+1}|1-i\tau H/\hbar|q_j\rangle=\delta(q_{j+1}-q_j)-i\tau\hbar\langle q_{j+1}|H|q_j\rangle\\
&=& \frac{1}{2\pi\hbar}\int dp\ \text{Exp}[\frac{ip}{\hbar}(q_{j+1}-q_j)]-\frac{i\tau}{\hbar}\langle q_{j+1}|H|q_j\rangle 
\end{eqnarray}	
Si asumimos que el Hamiltoniano es una función de $p$ y $q$de la forma: $H=\frac{p^2}{2m}+V(q)$, entonces:
\begin{eqnarray}
\nonumber \langle q_{j+1}|\frac{p^2}{2m}|q_j \rangle &=& \int dpdp\prime \langle q_{j+1}|p\prime\rangle\langle p\prime|\frac{p^2}{2m}|p\rangle \langle p|q_j \rangle\\
\nonumber &=&\int \frac{dpdp\prime}{2\pi\hbar}\ \text{Exp}[\frac{i}{\hbar}(p\prime q_{j+1}-pq_j)]\frac{p^2}{2m}\delta(p-p\prime)\\
&=& \int \frac{dp}{2\pi\hbar}\ \text{Exp}[\frac{i}{\hbar}p(q_{j+1}-q_j)]\frac{p^2}{2m}
\end{eqnarray}
De una manera similar:
\begin{eqnarray}
\nonumber \langle q_{j+1}|V(q)|q_j \rangle &=& V(\frac{q_{j+1}+q_j}{2})\langle q_{j+1}|q_j\rangle\\
\nonumber &=& V(\frac{q_{j+1}+q_j}{2})\delta(q_{j+1}-q_j)\\
\langle q_{j+1}|V(q)|q_j \rangle &=& V(\frac{q_{j+1}+q_j}{2})\int \frac{dp}{2\pi\hbar}\ \text{Exp}[\frac{i}{\hbar}p(q_{j+1}-q_j)]
\end{eqnarray}
Por tanto $\langle q_{j+1}|H|q_j\rangle=\int \frac{dp}{h}\ \text{Exp}[\frac{i}{h}p(q_{j+1}-q_j)]H(p,q)$ y:
\begin{equation}
\langle q_{j+1}t_{j+1}|q_jt_j\rangle=\frac{1}{h}\int dp_j\ \text{Exp}[\frac{i}{\hbar}[p_j(q_{j+1}-q_j)-\tau H(p_j,q_j)]]
\end{equation}
Finalmente:
\begin{equation}
\langle q_{f}t_{f}|q_it_i\rangle=\lim_{N \to \infty}\int\prod_{j=1}^{N}dq_j\prod_{j=0}^{N}\frac{dp_j}{h}\ \text{Exp}\left\{ \frac{i}{\hbar}\sum_{j=0}^{N}[p_{j}(q_{j+1}-q_{j})-\tau H(p_{j},q_{j})]\right\}
\end{equation}
En el continuo:
\begin{equation}
\langle q_{f}t_{f}|q_it_i\rangle=\int\frac{\mathcal{D}q\mathcal{D}p}{h}\text{Exp\ensuremath{\left\{ \frac{i}{\hbar}\int_{t_{i}}^{t_{f}}(p\dot{q}-H(p,q))dt\right\} }}
\end{equation}
En el continuo $q$ se vuelve una función de $t$ y la integral anterior, es una integral funcional. La expresión (2.15) es la integral de trayectoria para la amplitud de transición entre $(q_i,t_i)$ y $(q_f,t_f)$. Esta integral se hace sobre todas las posibles trayectorias en el espacio de fase y $q(t), p(t)$ son funciones y no operadores, sin embargo es natural preguntarse por la convergencia de (2.15), esto es algo no trivial, sin embargo, de ahora en adelante asumiremos que esta integral existe y converge. 	
\\
Si el Hamiltoniano es tal que $H=\frac{p^2}{2m}+V(q)$
\begin{equation}
\langle q_{f}t_{f}|q_it_i\rangle=\lim_{N \to \infty}\int\prod_{j=1}^{N}dq_j\prod_{j=0}^{N}\frac{dp_j}{h}\ \text{Exp}\left\{ \frac{i}{\hbar}\sum_{j=0}^{N}[p_{j}(q_{j+1}-q_{j})-\tau \frac{p_j^2}{2m}-V(q)\tau]\right\}
\end{equation}
Y sabiendo que: $\int_{-\infty}^{\infty}\text{Exp}[-ax^{2}+bx+c]dx=\text{Exp}[\frac{b^{2}}{4a}+c](\frac{\pi}{a})^{\frac{1}{2}}$, entonces:
\begin{equation}
\langle q_{f}t_{f}|q_it_i\rangle=\lim_{N\to\infty}[\frac{1}{h}(\frac{\text{\ensuremath{\pi\hbar}2m}}{\tau})^{\frac{1}{2}}]^{N}\int\prod_{1}^{N}dq_{j}\text{Exp}\left\{ \frac{i}{\hbar}\sum_{0}^{N}\left((\frac{q_{j+1}-q_{j}}{\tau})^{2}\frac{m}{2}-V\right)\tau\right\} 
\end{equation}
En el continuo:
\begin{equation}
\langle q_{f}t_{f}|q_it_i\rangle=\mathcal{N}\int\mathcal{D}q\text{Exp\ensuremath{\left\{ \frac{i}{\hbar}\int_{t_{i}}^{t_{f}}\mathcal{L}(q,\dot{q})dt\right\} }}
\end{equation}
Donde $\mathcal{L}(q,\dot{q})$ es el lagrangiano clásico, sin embargo esto solo pasa si asumimos una forma específica del Hamiltoniano, cuando esto no es asi se tiene una acción efectiva. En teorñia de campos por ejemplo esta descomposición solo se puede hacer en el caso de campos Abelianos.
\newpage
 
 

\subsection{Teoría de Perturbaciones.}

Vamos a ilustrar cómo usar el método de integrales de trayectoria en procesos de dispersión, este tipo de procesos involucran la interacción de una partícula con un potencial $V(x)$. Debido a que no siempre podemos calcular analíticamente la integral (2.18) entonces es necesario acudir a la teoría de perturbaciones, esta es aplicable en el regimen en que la energía de interacción $E_I<\hbar$. En este caso podemos escribir:
\begin{equation}
\text{Exp}\left[\frac{-i}{\hbar}\int_{t_{i}}^{t_{f}}V(x,t)dt\right]=1-\frac{i}{\hbar}\int_{t_{i}}^{t_{f}}V(x,t)dt+\left(\frac{i}{\hbar}\right)^{2}\frac{1}{2!}\left[\int_{t_{i}}^{t_{f}}V(x,t)dt\right]^{2}+...
\end{equation}
Cuando reemplazamos esto en la expresión (2.18) obtenemos:
\begin{equation}
K=K_0+K_1+K_2+...\ \text{donde}\ K_0=\mathcal{N}\int \mathcal{D}x \text{Exp}\left[ \frac{i}{\hbar}\int \frac{1}{2}m\dot{x}^2dt\right] 
\end{equation}
Para calcular $K_0$, escribamos la forma discretizada de (2.20)
\begin{eqnarray}
\nonumber K_0&=&\lim_{n\to\infty}\left(\frac{m}{i\hbar\tau}\right)^{\left(\frac{n+1}{2}\right)}\int_{-\infty}^{\text{\ensuremath{\infty}}}\prod_{j=1}^{n}dx_{j}\text{Exp}\left[\frac{im}{2\hbar\tau}\sum_{j=0}^{n}\left(x_{j+1}-x_{j}\right)^{2}\right]\\
\nonumber &=& \lim_{n\to\infty}\left(\frac{m}{i\hbar\tau}\right)^{\left(\frac{n+1}{2}\right)}\frac{1}{\left(n+1\right)^{\frac{1}{2}}}\left(\frac{i\hbar\tau}{m}\right)^{\frac{n}{2}}\text{Exp}\left[\frac{im}{2\hbar(n+1)\tau}\left(x_{f}-x_{i}\right)^{2}\right]\\
K_0(x_ft_f,x_it_i)&=& \Theta(t_{f}-t_{i})\left(\frac{m}{i\hbar(t_{f}-t_{i})}\right)^{1/2}\text{Exp}\left[\frac{im}{2\hbar(t_{f}-t_{i})}\left(x_{f}-x_{i}\right)^{2}\right] 
\end{eqnarray}
Calculemos ahora $K_1$:
\begin{equation}
K_1=\lim_{n\to\infty}\frac{-i}{\hbar}N^{(n+1)/2}\sum_{i=1}^{n}\int\text{Exp}\left[\frac{im}{2\hbar\tau}\sum_{j=0}^{n}(x_{j+1}-x_{j})^{2}\right]V(x_{i},t_{i})dx_{1}...dx_{n}
\end{equation}
Ahora como solo $V(x)$ depende de $x_i$, separamos (2.22) como:
\begin{eqnarray}
\nonumber K_1&=&\lim_{n\to\infty}\frac{-i}{\hbar}\sum_{i=1}^{n}\left\{ \int N^{(n-i+1)/2}\text{Exp}\left[\frac{im}{2\hbar\tau}\sum_{j=i}^{n}(x_{j+1}-x_{j})^{2}\right]dx_{i}...dx_{n}\right\}\\
& &\times  \left\{ \int N^{i/2}\text{Exp}\left[\frac{im}{2\hbar\tau}\sum_{j=0}^{i-1}(x_{j+1}-x_{j})^{2}\right]dx_{1}...dx_{i-1}\right\} V(x_{i},t_{i})
\end{eqnarray}
Los dos términos en corchetes en (2.23) son $K_0(xt,x_ft_f)$ y $K_0(x_it_i,xt)$, así:
\begin{equation}
K_1(x_ft_f,x_it_i)= -\frac{i}{\hbar}\int_{t_{i}}^{t_{f}}dt\int_{-\infty}^{\infty}K_{0}(x_{f}t_{f}.xt)V(x,t)K_{0}(xt,x_{i}t_{i})dx
\end{equation}
Como $K_0(x_ft_f,xt)=0$ si $t>t_f$ y $K_0(xt,x_it_i)=0$ si $t<t_i$, entonces podemos escribir:
\begin{equation}
K_1(x_ft_f,x_it_i)=-\frac{i}{\hbar}\int_{-\infty}^{\infty}dt\int_{-\infty}^{\infty}K_{0}(x_{f}t_{f}.xt)V(x,t)K_{0}(xt,x_{i}t_{i})dx 
\end{equation}
De la misma manera se puede probar que:
\begin{equation}
K_2(x_ft_f,x_it_i)=\left(\frac{-i}{\hbar}\right)^2\int_{-\infty}^{\infty}dt_{1}dt_{2}dx_{1}dx_{2}K_{0}(x_{f}t_{f}.x_{2}t_{2})V_2K_{0}(x_{2}t_{2},x_{1}t_{1})V_1K_{0}(x_{1}t_{1},x_{i}t_{i})
\end{equation}
Esta es una solución en series para $K$ y recibe el nombre de Serie de Born, en la expresión general para $K_n$ no se tiene el factor $n!$ ya que hay ese mismo numero de formas para ordenar los $n$ potenciales $V(x)$ que entran en el propagador.
\\
\\
Por último mostraremos que \textbf{el propagador es la función de Green de la ecuación de Schrödinger.} Para esto sustituyamos la expresión para la serie de Born en la ecuación (2.2):
\begin{eqnarray}
\nonumber \psi(\vec{x}_f,t_f)&=&\int K_{0}(\vec{x_{f}}t_{f},\vec{x_{i}}t_{i})\psi(\vec{x_{i}},t_{i})d\vec{x_{i}}\\
& & \frac{-i}{\hbar}\int K_{0}(\vec{x_{f}}t_{f},\vec{x}t)V(\vec{x},t)K_{0}(\vec{x}t,\vec{x_{i}}t_{i})\psi(\vec{x_{i}},t_{i})d\vec{x}dtd\vec{x_{i}}+O(\hbar^{2})
\end{eqnarray}
Hemos cambiado a tres dimensiones espaciales y los otros términos en la serie hacen converger el último $K_0$ a $K$, por tanto:
\begin{equation}
\psi(\vec{x}_f,t_f)=\int K_{0}(\vec{x_{f}}t_{f},\vec{x_{i}}t_{i})\psi(\vec{x_{i}},t_{i})d\vec{x_{i}}-\frac{i}{\hbar}\int K_{0}(\vec{x_{f}}t_{f},\vec{x}t)V(\vec{x},t)\psi(\vec{x},t)d\vec{x}dt
\end{equation}
Cuando $t_i \to -\infty$,no hay presencia de potencial por tanto $\psi$ se vuelve una onda plana. Así:
\begin{equation}
\psi(\vec{x}_f,t_f)=\phi(\vec{x}_ft_f)-\frac{i}{\hbar}\int K_{0}(\vec{x_{f}}t_{f},\vec{x}t)V(\vec{x},t)\psi(\vec{x},t)d\vec{x}dt
\end{equation}
Aplicando el operador $\hat{H}=\frac{\hbar^2}{2m}\nabla_{\vec{x}_ft_f}+i\hbar\frac{\partial}{\partial t_f}$ en la ecuación (2.29) y usando $\hat{H}\psi(x,t)=V(x,t)\psi(x,t)$:
\begin{eqnarray}
\nonumber \hat{H}(\psi(\vec{x}_f,t_f))&=&\hat{H}(\phi(\vec{x}_ft_f))-\frac{i}{\hbar}\int\hat{H} (K_{0}(\vec{x_{f}}t_{f},\vec{x}t))V(\vec{x},t)\psi(\vec{x},t)d\vec{x}dt\\
\nonumber V(\vec{x}_f,t_t)\psi(\vec{x}_f,t_f)&=&0-\frac{i}{\hbar}\int\hat{H} (K_{0}(\vec{x_{f}}t_{f},\vec{x}t))V(\vec{x},t)\psi(\vec{x},t)d\vec{x}dt
\end{eqnarray}
Por tanto:
\begin{equation}
\left(\frac{\hbar^2}{2m}\nabla_{\vec{x}_ft_f}+i\hbar\frac{\partial}{\partial t_f}\right)K_{0}(\vec{x_{f}}t_{f},\vec{x}t)=i\hbar\delta(\vec{x}-\vec{x}_f)\delta(t-t_f)
\end{equation}
Esto último era lo que queríamos probar.

\subsection{Matriz $\mathcal{S}$.}
En un experimento de dispersión es razonable suponer que las partículas al principio y al final del proceso son partículas libres, estas ondas planas están distribuidas en todo el espacio. Sin embargo esto último lleva a una contradicción ya que la presencia del centro dispersor no permite que en sus cercanias la solución sea una onda plana. Para resolver este inconveniente se puede proponer lo que se llama una fuente dinámica, que se "prenda y apague" lentamente tal que las partículas en los estados (final/incial) puedan ser libres y por tanto la aproximación de ondas planas sea válida en este regimen.\\



\section{El experimento de la doble rendija.}

\section{Campos escalares.}
\section{Campos fermiónicos.}
\section{Teorías Gauge y campos de Yang-Mills}
\section{La teoria de Yukawa}