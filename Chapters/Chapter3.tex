\chapter{Integrales de trayectoria en espacios curvos.}
Al nivel más fundamental la situación que nos presenta la fisica teórica es un modelo, con un alto grado de soporte experimental, en el cual solo hay tres interacciones: la cromo-dinámica cuaántica (QCD), que explica la interacción mediante la cual los quarks interactúan para formar los hadrones; la interacción débil (EW) que explica los procesos de decaimiento radiactivo y los fenomenos electromagneticos en una teoría unificada; y finalmente la gravedad. Las primeras dos interacciones se entienden en el contexto de la teoria cuántica de campos, en particular la teoría de campos Gauge, donde las interacciones se explican mediante el intercambio de bosones Gauge (gluones para QCD y el fotón y los bosones W,Z para EW). Sin embargo la teoría que mejor explica los fenómenos gravitatorios. la relativdad general (GR), es una teoría totalmente diferente debido a su naturaleza geométrica.
\\
\\
Sin embargo el objetivo de la física es reducir el numero de teorías, conceptos y esquemas al mínimo, es por esto que muchos físicos trabajan en teorías que unifiquen estos dos esquemas: supergravedad, teoría de supercuerdas, gravedad cuántica de lazos, etc.


	   	
\section{Teoría clásica de campos en espacios curvos.}

Quisque tristique urna in lorem laoreet at laoreet quam congue. Donec dolor turpis, blandit non imperdiet aliquet, blandit et felis. In lorem nisi, pretium sit amet vestibulum sed, tempus et sem. Proin non ante turpis. Nulla imperdiet fringilla convallis. Vivamus vel bibendum nisl. Pellentesque justo lectus, molestie vel luctus sed, lobortis in libero. Nulla facilisi. Aliquam erat volutpat. Suspendisse vitae nunc nunc. Sed aliquet est suscipit sapien rhoncus non adipiscing nibh consequat. Aliquam metus urna, faucibus eu vulputate non, luctus eu justo.

\subsection{Caso abeliano.}
\subsection{Caso No-abeliano.}
\subsection{La simetría de Lorentz como teoría Gauge.}
\subsection{La ecuación de Dirac en el espacio de Schwarzschild.}

\section{La ecuación de Schrodinger en espacio curvo.}

\section{Transformaciones de coordenadas.}
\section{El átomo de Hidrógeno vía integrales de trayectoria.}
\section{El propagador para el efecto Sagnac.}