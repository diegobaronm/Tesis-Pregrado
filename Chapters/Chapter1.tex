\chapter{Introducción}
En la actualidad la ciencia, especificamente la física teórica, nos presenta un panorama, respaldado por una gran cantidad de experimentos, en el cual podemos entender el mundo al nivel más fundamental de la siguiente manera: existen 3 tipos de interacciones fundamentales: la gravedad, la interacción electrodébil y la interacción fuerte. La gravedad es la interacción responsable de la aceleración que sufren los cuerpos debido a su contenido de masa-energía, la teoría de la relatividad general (GR) explica satisfactoriamente los fenómenos asociados a esta interacción, esta es una teoría clásica de campos donde las fuerzas gravitatorias son reemplazadas por el concepto de curvatura del espacio-tiempo[1]. La interacción electrodébil es responsable de los procesos de decaimiento radiactivo y de las interacciones electromagnéticas (esto incluye a todas las partículas que estén cargadas eléctricamente y también bajo isospín débil) y finalmente la interacción fuerte que es responsable del proceso de hadronización que sufren los quarks (todas las partículas con carga de color sufren esta interacción). Las teorías físicas que explican estas dos últimas interacciones son el modelo Weinberg-Salam-Glashow (o teoría electrodébil)(EW) y la cromodinámica cuántica (QCD) [2], estas dos son teorías cuánticas de campo.
\\
\\
Entre las muchas cuestiones que la física moderna no ha resuelto aún, se incluye la pregunta por la naturaleza cuántica de la gravedad, como ya se ha expuesto, la mejor teoría que tenemos para explicar los fenómenos gravitatorios es la GR y esta es una teoría clásica de campos, una teoría cuántica del campo gravitatorio quizás nos ayudaría a explicar fenómenos como la inflación [3], el origen del universo, el problema de las singularidades de espacio-tiempo, etc... Sin embargo una pregunta algo más modesta, que apunta en esta dirección y también es importante es saber cómo cambia la fenomenología de las teorias cuánticas que ya conocemos en la presencia de campos gravitatorios. Estas teórias se conoces como \textit{teorías cuánticas en espacio-tiempo curvo.}
\\
\\
Una forma de abordar los problemas de cuantización en espacios curvos es el formalismo de integrales de trayectoria, este trabajo es una recopilación de los métodos utilizados para resolver este tipo de cuestiones en problemas de primera y segunda cuantización. El trabajo está dividido en dos capítulos principales: el primero de ellos está dedicado a las integrales de trayectoria en el espacio plano (euclidéo y de Minkowski) con el fin de tratar problemas cuánticos de primera y segunda cuanización, el segundo es una recopilación de la técnica de integrales de trayectoria en espacios curvos para resolver problemas de primera cuantización.
\\
\\
El capítulo 2 empieza introduciendo el concepto mas fundamental del formalismo de integrales de trayectoria, \textit{el propagador}, luego de una discusión sobre su significado físico exploramos cual es la forma explícita del propagador para un partícula cuántica que obedece la ecuación de Schrödinger y desarrollamos algunas propiedades adicionales, como el tratamiento por teoría de perturbaciones del problema de primera cuantización, la definición de la amplitud de dispersión (matriz $\mathcal{S}$) y para finalizar la sección desarrollamos un preámbulo de algunas propiedades que ayudarán luego a introducir la técnica del propagador para problemas de muchos cuerpos en mecánica cuántica. En la sección  $\mathsection$2.2 resolvemos el famoso problema del experimento de Young con el método de integrales de trayectoria, para esto primero damos una forma cerrada del propagador para una partícula libre y luego estudiamos la amplitud de probabilidad para la difracción por una y dos rendijas.
\\
\\
Desde la  $\mathsection$2.3 nos dedicamos al estudio del método del propagador para problemas de segunda cuantización, es esta sección discutimos el caso del campo escalar.Se muestra cómo realizamos el cálculo del funcional generatriz, como de este podemos derivar las funciones de Green y para finalizar la sección se estudia el caso del campo escalar con términos de interacción. En la siguiente sección estudiamos como encontrar el funcional generatriz en el caso de campos fermiónicos de spín $\frac{1}{2}$ (campos de Dirac), se introduce el concepto de variables de Grassmann y como entran estas en una teoría cuántica de campos, en la parte final deducimos la que es conocida como la fórmula de la reducción, que tiene que ver con los elementos de la matriz $\mathcal{S}$ en el caso del problema de muchos cuerpos. La parte final del capítulo 2 se dedica a hacer una introducción a los metodos funcionales para campos de Yang-Mills, en esta sección estudiamos el método de Faddevv-Popov y obtenemos una fórmula explícita para el funcional generatriz de un campo Gauge.
\\
\\
En el capítulo 3 comenzamos con el problema de acoplar una teoría de campo clásica a la gravitación, esta primera sección pretende ser una motivación y en ella estudiamos como al intentar hacer invariante Gauge local una teoría de campos, terminamos por introducir un campo Gauge que se termina acoplando con el campo de materia para hacer invariante la teoría. Este procedimiento lo ejemplificamos en el caso de simetrías Gauge abelianas y no abelianas, para luego ver como al hacer la simetría de Lorentz una simetría local, terminamos por introducir de nuevo un campo Gauge, que en ese caso identificamos con la gravedad, para finalizar esta sección introductoria y no dejar todo en el aire, desarrollamos la ecuación de Dirac en el espacio de Schwarzschild. En la  $\mathsection$3.2 empezamos a estudiar el propagador para el caso de una partícula que obedece la ecuación de Schrödinger en espacio curvo, aquí se introduce el problema del ordenamiento, que aparece al intentar calcular el propagador y por tanto hablamos del ordenamiento de Weyl, esto nos permite calcular el propagador en espacio curvo y dar por terminada la sección.
\\
\\
En lo que sigue estudiamos la técnica de transformaciones de coordenadas en integrales de trayectoria, en la  $\mathsection$3.3 presentamos el formalismo necesario, estudiamos el caso de cambiar el propagador para una partícula libre a coordenadas polares y en la sección que sigue vemos la utilidad de este tipo de procedimiento al resolver el problema del átomo de Hidrógeno en dos dimensiones mediante el método del propagador. La sección final está dedicada a estudiar un problema que creemos que no está reportado anteriormente en la literatura y es el calculo del efecto Sagnac mediante el método de integrales de trayectoria. Primero, empezamos describiendo el efecto y calculando el corrimiento de lineas desde la teoría clásica, en este caso la relatividad especial, luego, en la sección final de este trabajo calculamos el propagador para este problema.
\\
\\
Este trabajo ha sido escrito con el proposito de recopilar de una manera ordenada y didáctica el producto de la investigación y lectura de información que hemos hecho durante los últimos 3 semestres de mi carrera con mi asesor, hemos puesto especial énfasis en mostrar los detalles matemáticos que presentaron algún tipo de dificultad durante el tiempo que hemos estado estudiando, el trabajo está dirigido para estudiantes de nivel de pregrado que quieran empezar desde cero a estudiar el formalismo de integrales de trayectoria y servirá tanto para quienes apunten en la dirección de física de partículas como para aquellos que buscan una perspectiva algo mas teórica y quieran aventurarse en el camino de las teorías cuánticas de campo en espacios curvos. Sin embargo, también hemos querido añadir a este trabajo una idea original, que debido a la gran cantidad de información que hemos tenido que explorar y a la confusión que en algún momento ha surgido, no es lo sufientemente extensa como queremos. Finalmente creemos que las limitaciones de esta publicación son de tiempo, pero creemos que este trabajo es como una buena entrada en un restaurante, que calma brevemente el apetito pero deja con ganas de más...