\chapter{Conclusiones.}
Es importante, con el fin de concretar las ideas que hemos expuesto, dar una serie de conclusiones finales: como hemos visto el formalismo de integrales de trayectoria puede ser usado para tratar gran variedad de fenómenos cuánticos, problemas de primera y segunda cuantización convencionales como la partícula libre, el experimento de Young y problemas algo más complicados como mecánica cuántica de muchos cuerpos. Este formalismo tiene la ventaja de permitirnos calcular las amplitudes de probabilidad cuántica, sin tener que hacer uso del formalismo de operadores, por el contrario conecta la acción clásica del problema que estemos intentando tratar con los aspectos cuánticos del mismo.
\\
\\
En el segundo capítulo de este trabajo nos hemos dedicado a mostrar como el formalismo de integrales de trayectoria permite calcular las funciones de Green en teoría cuántica de campos, para el caso de campos escalares, campos fermiónicos y campos Gauge, tanto para la teoría libre como para teorías con términos de interacción. Otra derivación importante que hemos hecho es el cálculo de los elementos de matriz de la matriz de dispersión, evitándonos el famoso teorema de Wick que aparece en el formalismo canónico, con todos estos resultados en principio estamos listos para calcular la sección eficaz de algún tipo de interacción como la teoría de Yukawa o la electrodinámica cuántica (QED).
\\
\\
Del tercer capítulo la moraleja puede ser la siguiente: al intentar tratar un problema cuántico vía integrales de trayectoria en espacio-tiempo curvo necesitamos primero estudiar como se ve el lagrangiano clásico de la teoría en espacio-tiempo curvo. Esto implica estudiar el famoso problema de espinores en espacio-tiempo curvo, si es que nuestro campo tiene spín semientero intrínseco, en caso de que estemos en presencia de un campo escalar o vectorial es más sencillo debido a que estos objetos tienen reglas de transformación bien definidas en espacio-tiempo curvo.
\\
\\
Luego nos hemos dado cuenta que al intentar calcular la integral de trayectoria en espacio curvo, el hecho de que el hamiltoniano sea sensible al ordenamiento de operadores introduce términos de corrección proporcionales a $\hbar^2$ y a factores geométricos. Otra característica importante es que dependiendo del tipo de ordenamiento escogido, los términos de corrección toman una forma funcional diferente. Adicionalmente hemos mostrado como realizar transformaciones de coordenadas en los propagadores y cómo este tipo de procedimientos han hecho posible la resolución de problemas como el átomo de Hidrógeno en el formalismo del integrales de trayectoría.
\\
\\
Finalmente hemos mostrado que efectos que suceden en espacio-tiempos no triviales, como el efecto Sagnac, pueden ser explicados usando el método del propagador en espacio-tiempo curvo. En esta última sección hemos mostrado como utilizar la maquinaria desarrollada durante todo el trabajo para tratar un problema realista. Es importante decir que este trabajo no está acabado y sería muy interesante perfilar nuestros esfuerzos en la dirección de usar ecuaciones como la de Dirac en espacio-tiempo curvo, e intentar de nuevo calcular el efecto Sagnac, más ambicioso aún, sería intentar explorar como acoplar estos métodos a problemas de segunda cuantización.